\documentclass[]{book}
\usepackage{lmodern}
\usepackage{amssymb,amsmath}
\usepackage{ifxetex,ifluatex}
\usepackage{fixltx2e} % provides \textsubscript
\ifnum 0\ifxetex 1\fi\ifluatex 1\fi=0 % if pdftex
  \usepackage[T1]{fontenc}
  \usepackage[utf8]{inputenc}
\else % if luatex or xelatex
  \ifxetex
    \usepackage{mathspec}
  \else
    \usepackage{fontspec}
  \fi
  \defaultfontfeatures{Ligatures=TeX,Scale=MatchLowercase}
\fi
% use upquote if available, for straight quotes in verbatim environments
\IfFileExists{upquote.sty}{\usepackage{upquote}}{}
% use microtype if available
\IfFileExists{microtype.sty}{%
\usepackage[]{microtype}
\UseMicrotypeSet[protrusion]{basicmath} % disable protrusion for tt fonts
}{}
\PassOptionsToPackage{hyphens}{url} % url is loaded by hyperref
\usepackage[unicode=true]{hyperref}
\hypersetup{
            pdftitle={Earth Observation for Quantifying Urban Growth and its Application to Sustainable City Development},
            pdfauthor={Submitted in accordance with the requirements for the degree of Doctor of Philosophy; School of Geography and Environmental Science,; University of Southampton},
            pdfborder={0 0 0},
            breaklinks=true}
\urlstyle{same}  % don't use monospace font for urls
\usepackage{natbib}
\bibliographystyle{apalike}
\usepackage{longtable,booktabs}
% Fix footnotes in tables (requires footnote package)
\IfFileExists{footnote.sty}{\usepackage{footnote}\makesavenoteenv{long table}}{}
\usepackage{graphicx,grffile}
\makeatletter
\def\maxwidth{\ifdim\Gin@nat@width>\linewidth\linewidth\else\Gin@nat@width\fi}
\def\maxheight{\ifdim\Gin@nat@height>\textheight\textheight\else\Gin@nat@height\fi}
\makeatother
% Scale images if necessary, so that they will not overflow the page
% margins by default, and it is still possible to overwrite the defaults
% using explicit options in \includegraphics[width, height, ...]{}
\setkeys{Gin}{width=\maxwidth,height=\maxheight,keepaspectratio}
\IfFileExists{parskip.sty}{%
\usepackage{parskip}
}{% else
\setlength{\parindent}{0pt}
\setlength{\parskip}{6pt plus 2pt minus 1pt}
}
\setlength{\emergencystretch}{3em}  % prevent overfull lines
\providecommand{\tightlist}{%
  \setlength{\itemsep}{0pt}\setlength{\parskip}{0pt}}
\setcounter{secnumdepth}{5}
% Redefines (sub)paragraphs to behave more like sections
\ifx\paragraph\undefined\else
\let\oldparagraph\paragraph
\renewcommand{\paragraph}[1]{\oldparagraph{#1}\mbox{}}
\fi
\ifx\subparagraph\undefined\else
\let\oldsubparagraph\subparagraph
\renewcommand{\subparagraph}[1]{\oldsubparagraph{#1}\mbox{}}
\fi

% set default figure placement to htbp
\makeatletter
\def\fps@figure{htbp}
\makeatother

\usepackage{booktabs}
\usepackage{amsthm}
\makeatletter
\def\thm@space@setup{%
  \thm@preskip=8pt plus 2pt minus 4pt
  \thm@postskip=\thm@preskip
}
\makeatother
\usepackage{etoolbox}
\makeatletter
\providecommand{\subtitle}[1]{% add subtitle to \maketitle
  \apptocmd{\@title}{\par {\large #1 \par}}{}{}
}
\makeatother

\title{Earth Observation for Quantifying Urban Growth and its Application to
Sustainable City Development}
\providecommand{\subtitle}[1]{}
\subtitle{Andrew MacLachlan}
\author{Submitted in accordance with the requirements for the degree of Doctor
of Philosophy \and School of Geography and Environmental Science, \and University of Southampton}
\date{September 2018}

\begin{document}
\maketitle

{
\setcounter{tocdepth}{1}
\tableofcontents
}
\section*{Abstract}\label{abstract}
\addcontentsline{toc}{section}{Abstract}

Urban areas are predicted to triple by 2030 in accommodating 68\% of the
global population, with anthropogenic landscape modifications to
impervious surfaces established as a critical driving force in local and
global climate change. Accurate temporal monitoring of urban expansion
and subsequent environmental issues are essential for ensuring the
future sustainability of our cities. In particular the urban heat island
effect is considered one of the major problems posed to humans in the
21st century associated with detrimental health impacts and increased
energy demand, emissions, water output and economic expenditure. Yet,
alongside a uniform modelling omission from global climate models,
mitigation of the urban heat island effect lacks a global standardised
framework, representative data for modelling impacts, and robust
academic outputs for policy incorporation. These limitations are
precluding effective data-informed governance.

This thesis presents a holistic policy-applicable approach for
accurately monitoring and sustainably planning (re)development in
relation to metropolitan and local level urban temperature dynamics.
This is achieved through generating land cover maps from Earth
observation data using a temporally consistent methodology with
refinements to urban estimates based upon comparison to high resolution
imagery. Variations between changes in land cover and land surface
temperature are determined at the metropolitan level to aid sustainable
urban growth plans. Temperature is then minimised at the local level
through a modelling approach to optimally place vegetation with a
proposed new development. The application area for this thesis is the
Perth Metropolitan Region in Western Australia which has experienced
sustained outward, non-strategic and low density expansion in response
to booming natural resource sector. The presented methodology makes
progress to aligning urban heat island mitigation efforts with global
targets including the United Nation's Sustainable Development Goals and
New Urban Agenda, providing a reproducible method, transferable to other
global metropolitan regions to improve sustainable city planning.

\chapter*{Acknowledgements}\label{acknowledgements}
\addcontentsline{toc}{chapter}{Acknowledgements}

\section*{Personal}\label{personal}
\addcontentsline{toc}{section}{Personal}

I wish to sincerely thank both my supervisors Dr Biggs and Dr Roberts
alongside my advisor Dr Boruff for their advice, contributions and
enthusiasm throughout this thesis. Specifically, Dr Biggs continuously
challenged how developed methodologies were applicable in applied
planning terms, complemented by Dr Roberts who would challenge more
technical aspects of the research and Dr Boruff who would often question
the applicability to the bigger, global picture. Combined they have been
a simply fantastic supervisory team that have undoubtedly developed my
overall research skills. Beyond the methodological and technical aspects
of the project the team have consistently encouraged me, provided
numerous further opportunities and promoted both my work and I. I am
extremely grateful the time all three academics have invested in me and
for how much I have learnt from them over the last four years. I also
thank those not officially involved with this thesis but have
nonetheless contributed in ways they were probably unaware of. This is
especially true of all those I have cycled with, taking my mind away
from research, subsequently feeling refreshed and often having
methodological breakthroughs. Similarly (soon to be Drs) Luke Brown and
Micheline Campbell for their friendship, advice and support through the
rollercoaster of emotions during the project. I thank Amelia Vincent for
her unrelenting encouragement and empathy over the last four years.
Finally I wish to thank my parents, Valerie and Angus MacLachlan, for
their continued encouragement and investment throughout my entire
education.

\section*{Funding and data access}\label{funding-and-data-access}
\addcontentsline{toc}{section}{Funding and data access}

This work was supported by the Economic and Social Research Council
(ESRC) {[}grant number ES/J500161/1{]} and the University of Southampton
School of Geography and Environmental Science, with institutional visit
funding support from the University of Southampton via the World
University Network (WUN) researcher mobility programme and Study Abroad
department. Appreciation is shown to the University of Western Australia
(UWA) and the UWA School of Agriculture and Environment for supporting
grant and visa applications alongside hosting institutional visits.
Access to Urban Monitor high resolution data over the Perth Metropolitan
Region was provided in part through the National Environmental Science
Program, Clean Air and Urban Landscapes Hub. I would like to
respectively thank t the United States Geological Survey (USGS) and
Goddard Space Flight Center for continuing to supply free Landsat and
Moderate Resolution Imaging Spectroradiometer (MODIS) data used
throughout this project. I also express gratitude to Western Australia
Department of Planning (WAPC), in particular Matt Devlin and Lisl van
Aarde for urban estimate data and supplementary planning information
used within chapter 4. In a similar theme the City of Fremantle,
especially Mayor Brad Pettitt, Paul Garbett, Marcel Maron and Gavin
Giles who provided valuable localised operational planning insights used
to inform analysis in chapter 7. I also extend further thanks to the
ESRC, School of Geography and Environmental Science and University of
Southampton for the numerous free training sessions and qualifications
on offer that have enhanced my research outputs, professional
development and overall doctorate experience. Finally I wish to thank my
mother and late father for continuing to support my education and
providing funds for research equipment and field visits.

\chapter*{Declaration of Authorship}\label{declaration-of-authorship}
\addcontentsline{toc}{chapter}{Declaration of Authorship}

I, Andrew Charles MacLachlan

declare that this thesis and the work presented in it are my own and has
been generated by me as the result of my own original research.

Earth Observation for Quantifying Urban Growth and its Application to
Sustainable City Development

I confirm that:

\begin{enumerate}
\def\labelenumi{\arabic{enumi}.}
\item
  This work was done wholly or mainly while in candidature for a
  research degree at this University;
\item
  Where any part of this thesis has previously been submitted for a
  degree or any other qualification at this University or any other
  institution, this has been clearly stated;
\item
  Where I have consulted the published work of others, this is always
  clearly attributed;
\item
  Where I have quoted from the work of others, the source is always
  given. With the exception of such quotations, this thesis is entirely
  my own work;
\item
  I have acknowledged all main sources of help;
\item
  Where the thesis is based on work done by myself jointly with others,
  I have made clear exactly what was done by others and what I have
  contributed myself;
\item
  Parts of this work have been published as:
\end{enumerate}

MacLachlan, A., Biggs, E., Roberts, G. and Boruff, B., 2017. Urban
growth dynamics in Perth, Western Australia: using applied remote
sensing for sustainable future planning. Land, 6(1), p.9. {[}open
access{]}.

MacLachlan, A., Biggs, E., Roberts, G. and Boruff, B., 2017. Classified
Earth observation data between 1990 and 2015 for the Perth Metropolitan
Region, Western Australia using the Import Vector Machine algorithm,
PANGAEA, \url{https://doi.org/10.1594/PANGAEA.871017}
\protect\hyperlink{data}{data}. {[}open access{]}.

MacLachlan, A., Roberts, G., Biggs, E. and Boruff, B., 2017. Subpixel
land-cover classification for improved urban area estimates using
Landsat. International Journal of Remote Sensing, 38(20), pp.5763-5792.
{[}open access{]}.

MacLachlan, A., Biggs, E., Roberts, G. and Boruff, B., 2017.
Urbanisation-Induced Land Cover Temperature Dynamics for Sustainable
Future Urban Heat Island Mitigation. Urban Science, 1(4), p.38. {[}open
access{]}.

MacLachlan, A., Biggs, E., Roberts, G. and Boruff, B., 2018. Earth
Observation for Sustainable City Planning. {[}submitted to Global
Environmental Change{]}.

Smith, A., MacLachlan, A., Haworth, B., Biggs, E. and Maginn, P., 2018.
Demonstrating the global potential of a high-resolution spatiotemporal
population modelling framework. International Journal of Geographical
Information Science {[}under review{]}.

In all jointly published work presented in this thesis Andrew MacLachlan
was responsible for sourcing data, liaising with planning authorities,
devising methodological approaches, all analysis and producing and
revising manuscript drafts. Dr Biggs and Dr Roberts contributed ideas,
comments and data in a supervisory capacity based on their own expertise
throughout the research. Dr Boruff supported the research in similar
manner through an advisory role at the University of Western Australia.
Author attribution statements for each individual paper are presented
below, providing specific details into Andrew MacLachlan's
responsibilities and the supervisory and advisory support supplied by Dr
Biggs, Dr Roberts and Dr Boruff. All authors declare no completing
interests.

\section*{Author attribution
statements}\label{author-attribution-statements}
\addcontentsline{toc}{section}{Author attribution statements}

\subsection*{Chapter 4, paper 1a}\label{chapter-4-paper-1a}
\addcontentsline{toc}{subsection}{Chapter 4, paper 1a}

Chapter 4 corresponds to the publication:

MacLachlan, A., Biggs, E., Roberts, G. and Boruff, B., 2017. Urban
growth dynamics in Perth, Western Australia: using applied remote
sensing for sustainable future planning. Land, 6(1), p.1-14.

Andrew MacLachlan liaised with and obtained data from the Western
Australian Department of Planning, analysed all data and produced
manuscript drafts. The co-authors assisted in guiding the overall
research objectives, provided methodological advice and revised
manuscript text in a supervisor capacity. In particular Dr Biggs
contributed valuable figure and policy recommendations whilst Dr Roberts
ensured valid data normalisation suggesting analysis found in the
supplementary material, section 4.7.

\subsection*{Chapter 4, paper 1b}\label{chapter-4-paper-1b}
\addcontentsline{toc}{subsection}{Chapter 4, paper 1b}

The data produced using the methodology described in Paper 1a (main body
of chapter 4) corresponds to the publication:

MacLachlan, A., Biggs, E., Roberts, G. and Boruff, B., 2017. Classified
Earth observation data between 1990 and 2015 for the Perth Metropolitan
Region, Western Australia using the Import Vector Machine algorithm,
PANGAEA, \url{https://doi.org/10.1594/PANGAEA.871017}. Andrew MacLachlan
produced and formatted the data for publication, with methodological
advice from co-authors in a supervisory capacity.

\subsection*{Chapter 5, paper 2}\label{chapter-5-paper-2}
\addcontentsline{toc}{subsection}{Chapter 5, paper 2}

Chapter 5 corresponds to the publication:

MacLachlan, A., Roberts, G., Biggs, E. and Boruff, B., 2017. Subpixel
land-cover classification for improved urban area estimates using
Landsat. International Journal of Remote Sensing, 38(20), pp.5763-5792.

Andrew MacLachlan was responsible for designing the methodology,
processing all data and producing manuscript drafts. Co-authors agreed
upon and guided the research questions, provided access to high
resolution aerial data, gave methodological and practical advice and
reviewed manuscript drafts in a supervisory capacity. More specifically,
Dr Boruff provided high resolution data, with Dr Roberts advising Andrew
MacLachlan on the best Object Based Image classification procedures.

\subsection*{Chapter 6, paper 3}\label{chapter-6-paper-3}
\addcontentsline{toc}{subsection}{Chapter 6, paper 3}

Chapter 6 corresponds to the publication:

MacLachlan, A., Biggs, E., Roberts, G. and Boruff, B., 2017.
Urbanisation-Induced Land Cover Temperature Dynamics for Sustainable
Future Urban Heat Island Mitigation. Urban Science, 1(4), pp.1-21.

Andrew MacLachlan designed the methodological procedure, processed and
analysed all data and produced manuscript drafts. The co-authors
assisted in guiding research objectives and provided valuable advice and
manuscript comments in a supervisory capacity. Dr Biggs critically
commented on policy applications and data output, especially Figure 6 8
that was selected for the Journal Issue front cover. Dr Roberts provided
practical methodological advice and assisted in selecting appropriate
data for figures and tables presented throughout the paper.

\subsection*{Chapter 7, paper 4}\label{chapter-7-paper-4}
\addcontentsline{toc}{subsection}{Chapter 7, paper 4}

Chapter 7 corresponds to the pending publication:

MacLachlan, A., Biggs, E., Roberts, G. and Boruff, B., 2018. Earth
Observation for Sustainable City Planning {[}submitted to Global
Environmental Change{]}.

Andrew MacLachlan identified and processed all data required used within
the modelling methodology, produced figures, liaised with the City of
Fremantle and drafted all versions of the manuscript. Co-authors guided
the overall objectives ensuring linkage to previous work whilst also
providing critical feedback in a supervisory capacity. Dr Biggs ensured
policy relevance and critically commented on produced figures. Dr
Roberts contributed ideas throughout and reviewed the final draft
manuscript alongside Dr Boruff.

Signed: {[}signature removed from digital version{]} Date: 14/09/2018
Andrew MacLachlan

As a supervisor for this postgraduate student I certify that the
authorship attribution statements above are true and accurate.

Signed: {[}signature removed from digital version{]} Date: 14/09/2018 Dr
Biggs

As a supervisor for this postgraduate student I certify that the
authorship attribution statements above are true and accurate.

Signed: {[}signature removed from digital version{]} Date: 14/09/2018 Dr
Roberts

\chapter{Introduction}\label{introduction}

\section{An expanding urban area}\label{an-expanding-urban-area}

Urban areas only cover 0.5\% of Earth's terrestrial surface and yet are
one of the fastest growing land use types on a per area basis (Gashu and
Egziabher, 2018; Powell et al., 2007; Sexton et al., 2013). Population
growth is fuelling urbanisation with 55\% of the planet's 7.6 billion
people residing in urban areas in 2018, and which is expected to
accommodate up to 68\% of the global population by 2050 as a result of
an estimated additional 2.5 billion people (Powell and Roberts, 2010;
Sexton et al., 2013; Sharifi and Lehmann, 2014; United Nations, 2018).
Land use and land cover change represent the main driving force in
global environmental change, especially anthropogenic modifications from
vegetated to impervious surface material, through which water does not
infiltrate (Sundarakumar et al., 2012; Tan et al., 2009; Verburg et al.,
2015). In order to accommodate the expected increase in population,
urban area is predicted to triple by 2030 based on current trends,
increasing total global coverage to 0.9\% of Earth's terrestrial surface
(2000 baseline of 652,825 km2) (Seto et al., 2012).

Urban growth, defined as the sum of the increase in developed land
between two or more time periods, was traditionally thought of as an
intrinsic process and metric of economic success (Sundarakumar et al.,
2012). However, the benefits have now been carefully evaluated in
conjunction with any adverse social, economic and environmental impacts.
For example, continued outward expansion exponentially decreases
resource efficiency due to vast infrastructural requirements, additional
commuting time and household transportation expenditure (Batty et al.,
2003; Downs, 2005). The alteration of natural land to impervious
surfaces frequently induces the Urban Heat Island (UHI) effect whereby
man-made urban areas obtain comparatively higher atmospheric and surface
temperature than rural areas; the UHI effect is considered one of the
major environmental problems posed to humans in the 21st century (Rizwan
et al., 2008). In conjunction with expanding global urban areas and
population growth the extent and magnitude of the UHI will continue to
grow (Zhang et al., 2013). Furthermore, urbanisation can lead to
income-based neighbourhood segregation which exacerbates social and
ethnic divisions (Batty et al., 2003), and coupled with the UHI effect
can result in sociodemographic-driven healthcare implications during
heatwaves (Gronlund et al., 2015; Yu et al., 2010). Whilst urban
expansion can generate benefits, accurate and timely information on the
characteristics and associated impacts of urban expansion are critical
for assessing current and future needs regarding urban growth and
developing policy priorities for inclusive and sustainable
{[}socioeconomic and environmental{]} development, as reflected in
recent global initiatives e.g.~the United Nations (UN) New Urban Agenda
(UN-Habitat III, 2017), City Resilience Framework (CRF) (ARUP and The
Rockefeller Foundation, 2015) and 2030 Sustainable Development Goals
(SDGs) (Osborn et al., 2015). \textbf{The overall aim of this research
is to demonstrate the application of EO data in quantifying urban growth
and its impact on the UHI in order to illustrate its potential for
informing both global and metropolitan sustainable city development
goals.}

\section{Defining urban area}\label{defining-urban-area}

The term `urban area' is surrounded by conceptual vagueness, with
definitions varying based on discipline origin e.g.~academia, census
bureaus and metropolitan development agencies (Bennett, 2001). Weeks
(2010) defined `urban' as a characteristic of place whereby a spatially
concentrated population are organised around non-agricultural
activities. This definition is composed of (i) population size, (ii)
land area, (iii) population density, and (iv) economic and social
organisation. However, due to increased population movement towards
traditional urban centres and subsequent outward expansion, the
urban-rural divide is becoming difficult to differentiate (United
Nations, Department of Economic and Social Affairs, 2014; United
Nations, 2016; Weeks, 2010). In an effort to mitigate the increasing
pressure from rural-urban migration on city expansion and metropolitan
services developing countries have initiated schemes to attract urban
infrastructure to previously considered rural areas (Weeks, 2010). For
example, the Vietnamese government initiated side-line productions with
the slogan ``leaving the land without leaving the village'' (Rigg, 1998;
Weeks, 2010). Similarly, China's Special Economic Zones (SEZs), offering
corporations economic incentives, attempted to resolve socio-economic
problems associated with the centrally administrated system whilst also
importing a foreign knowledge base (Kam Ng and Tang, 2004). In 1980 the
city of Shenzhen was designated as China's first SEZ, originally
considered a tiny rural town located at the northern edge of Hong Kong
with a population of only 0.3 million. By 2000 SEZ status resulted in a
23 fold population increase to an estimated 7 million (Kam Ng and Tang,
2004; Zhou et al., 2016). Nevertheless it is difficult to exactly
determine at what point in time Schenzen's status changed from rural to
urban due to the multitude of possible factors that could be considered
(e.g.~population, land area or economic organisation). Statistical
bureaus define urban in various ways to ensure data availability for
specific applications. For example, the Australian Bureau of Statistics
(ABS) currently provide five definitions of urban area based upon
differing Statistical Areas (SA) levels of collection. Due to the
complexity of the ABS urban definitions they are summarised in Figure 1
1 (ABS, 2017). One important consideration within this framework is that
the Greater Capital City Statistical Areas (GCCSAs) have only been in
force since the 2011 census, replacing the statistical divisions that
previously covered the greater metropolitan area, often referred to as
the metropolitan region of each capital city constrained by Local
Government Areas (LGA); consequently no previous census data is
available for the complete GCCSAs (ABS, 2017).

Figure 1 1. Varying urban definitions associated with different datasets
produced by the Australian Bureau of Statistics (ABS). Adapted from ABS
(2017).

Similarly the United States of America (USA) Census Bureau provides
further urban definition variation (Figure 1 2). The USA Census Bureau
base urban categorisation on census blocks, tracts, cities and counties.
Census blocks are the smallest geographic census area, established from
visible features such as roads, property boundaries and rivers. Census
tracts are representative of neighbourhoods as defined by the Census
Bureau, generally obtaining between 2,500 and 8,000 people, with
boundaries also defined on visible features. Cities and associated
spatial boundaries are defined in legislation by the city government
whilst Counties are based upon geographic, political and administrative
subdivision of a State (U.S. Census Bureau, 2018).

Figure 1 2. Varying urban definitions associated with different datasets
produced by the United States of America (USA) Census Bureau. Adapted
from U.S. Census Bureau (2018).

In contrast the United Kingdom (UK) Office for National Statistics (ONS)
rather more simplistically define urban as settlements with a population
of 10,000 or more residents. Settlements are established from a grid of
hectare cells (100m x 100m) over England and Wales, with residential
properties inferred from Royal Mail's (UK's postal service) postcode
address file assigned to cells. Residential density for a set of
increasing radii surrounding each cell is calculated and compared to
defined standards assigning each cell to a settlement type
(e.g.~village, town, urban fringe). Census Output Areas (OA), the
smallest level of census data are combined with the classified cell
output and defined urban if both classed as a settlement and obtaining a
population of 10,000 or more (Bibby and Brindley, 2013).

These contrasting urban definitions and associated data are based upon
local governments and organisations being able to best identify features
that differentiate urban and rural (Brockerhoff, 2000). Nevertheless
individual country specific approaches hinder cross country comparison
and accurate global estimates due to the multitude of differing factors
that are considered in determining urban extent.

Urban development strategies that outline future development and
sustainability targets are frequently provided by metropolitan
development agencies such as the Spatial Development Strategy For London
(Mayor of London, 2016a), Perth's Directions 2031 (Western Australian
Planning Commission, 2010a) and Johannesburg's Spatial Development
Framework 2040 (City of Johannesburg Metropolitan Municipality, 2016).
These documents often fail to definitively define the term urban,
generating inconsistencies throughout national agencies. In the case of
Perth, Western Australia (WA), initial urban estimates only covered the
Perth Metropolitan Region (PMR) based upon cadastral land parcel
valuations, with later estimates considering spatial modelling, multiple
urban categories and the larger spatial extent of the Perth GCSSA
(Western Australian Planning Commission, 2010a). Consequently, whilst
the spatial extent of the defined metropolitan region or GCCSA will
align with the ABS extents, urban estimations are derived from differing
metrics.\\
The main overall limitation of the aforementioned classifications
pertains to the assumption of a dichotomous urban and rural divide.
However, due to the wide variations in urban designation, the UN
Population Division (1950) originally proposed an urban-rural continuum
(Weeks, 2010). Nevertheless, of the 228 countries for which the UN
compiles data, around half define urban extent based on administrative
boundaries, 51 use population size or density, 39 implement financial
metrics, 22 have no definition and 8 define all (e.g.~Singapore) or none
(e.g.~Polynesian countries) as urban (Brockerhoff, 2000). A lack of
consistent national and international urban definitions precludes
scientifically valid comparisons (Brockerhoff, 2000).

The advent of remote sensing provides an alternative approach for
mapping urban areas. Sensors on board satellites provide consistent and
reliable collection of data that when combined with approaches for
classification can determine urban extent over time based upon unique
and specific land cover characteristics. For this research urban is
defined as a determinant of urban land cover as calculated from the
spectral reflectance properties of temporally consistent user defined
areas (termed Regions Of Interest (ROIs)) from satellite Earth
Observation (EO) sensors: Landsat 5 Thematic Mapper (TM), and Landsat 8
Operational Land Imager (OLI). The reflected radiation monitored in each
spectral band was combined in producing unique surface reflectance
spectral signature for the ROIs subsequently matched to other similar
pixels through classification methodologies. Due to the heterogeneity of
urban areas and spectral confusion of classification methodologies
associated with spectrally similar land cover classes (e.g.~bare earth
and urban) urban land cover was classified as either high or low urban
albedo (Chen et al., 2014). The former represents surfaces with higher
solar reflectivity (e.g.~concrete) usually found in city centres whilst
the latter denotes lower solar reflectivity (e.g.~asphalt) commonly used
in residential developments (Yang et al., 2015).

In this sense urban area depicts land cover representing the biophysical
attributes of Earth's surface as opposed to land use that defines the
human purpose or intent applied to the biophysical attributes (Lambin et
al., 2001). Whilst this definition excludes population, economic and
social organisation factors, it provides the longest temporal record of
land cover derived from a replicable and temporally consistent
methodology appropriate in determining further environmental
relationships such as the association with temperature; important for
planning (sustainable) development. Within this research development
refers to an expansion of urban area based upon a change in land cover
whilst sustainable is defined as actions causing minimal (or no) damage
to the environment and humans enabling long-term continuation aligning
with the UN's combined sustainable development definition of development
``that meets the needs of the present without compromising the ability
of future generations to meet their own needs'' (Brundtland, 1987,
p.41).

\section{Types of urban expansion}\label{types-of-urban-expansion}

\subsection{Urban growth}\label{urban-growth}

Within scientific literature urban growth is often conceptualised using
a form of geographic data resulting in three main categories considering
the relationship to existing urban areas, namely infill, expansion and
outlying, with outlying obtaining three further sub categories of
isolated, linear branch and clustered branch (Wilson et al., 2003; Yuan
et al., 2005) (Figure 1 3). Infill growth is defined by the
(re)development of a small land tract surrounded by urban land cover.
Often characterised by the conversion of a non-developed area (e.g.~30
m2) to an urban surface with at least 40\% of the surrounding area that
is also defined as developed, usually occurring in areas of existing
infrastructure (Bhatta, 2010; Wilson et al., 2003). Expansion growth
constitutes conversion of non-developed area to developed, whilst being
surrounded by no more than 40\% existing developed area (Bhatta, 2010;
Wilson et al., 2003). The overall limiting factor of these
characterisations pertains to the assumption of conversion from
non-developed land (known as a greenfield site), when other urban land
cover could be converted from dormant use (e.g.~wasteland or abandoned
facilities, known as a brownfield site) to a form infill development.
However, this type of conversion is difficult to determine without land
use information accompanying geographic data. Outlying growth occurs
beyond the existing developed area, in the isolated form it is
characterised by one or several developed areas surrounded by minimal
developed land. The linear branch form refers to a new linear
development (e.g.~roads) that are surrounded by non-developed areas.
Finally the clustered branch form identifies large compact developments
that are neither isolated nor linear in nature (Wilson et al., 2003).

Figure 1 3. Forms of urban growth as defined by scientific literature.
Adapted from Bhatta (2010).

In contrast to academic literature, previously presented metropolitan
level policy documents in section 1.2 stipulate infill targets and
aspirations but often omit urban growth definitions providing no further
subcategories (e.g.~Perth and London) or vague definitions
(e.g.~Johannesburg) that are difficult to quantitatively assess. For
example, the Johannesburg Metropolitan Municipality specified a linear
future development scenario defined based on concentration of population
and jobs along extensive transit corridors, but lack a measureable
metric that would provide adequate monitoring. Whilst the Western
Australian Planning Commission (WAPC) aim to achieve 47\% of future
development as infill by 2050, monitoring considered zoned and not
necessarily developed industrial, commercial and residential land use
defined as infill or greenfield based upon residential density above or
below an unspecified threshold using census data (Western Australian
Planning Commission, 2016a). In Australia, census data is collected
every 5 years, restricting temporal monitoring of development and
assuming that changes in density across a range of land use types that
may not yet be developed is an accurate representation of infill
development (Western Australian Planning Commission, 2016a). The
vagueness associated with current policy definitions initiates
difficulties for local governments in meeting specified targets and in
the provision of policy-focused methodologies to accurately monitor
urban development.

\subsection{Urban sprawl}\label{urban-sprawl}

The expansion of low density suburbs into previously rural areas creates
exurbs, urban or suburban areas and raises the notion of urban sprawl
(Sun et al., 2013; Yuan et al., 2005). Currently there is no universally
accepted definition of urban sprawl, with multiple studies using urban
growth, urban sprawl, peri-urbanisation and exurban development
interchangeably without providing definitive definitions (Schneider,
2012; Schneider and Mertes, 2014; Suarez-Rubio et al., 2012; Xian et
al., 2012). Whilst Yuan et al. (2005) highlighted the requirement of an
undeveloped land buffer between established area and new development,
Castrence et al. (2014) simply stated urbanisation is a loss of open
space and agricultural land that leads to urban sprawl. In terms of
previously considered metropolitan policy documentation only
Johannesburg provided a definitive sprawl definition revolving around
dispersed over focused developed forms, whereas Perth's implied low
density, greenfield detached housing dwellings located on the urban
fringe and London's suggested development beyond the greenbelt.
Consequently in a similar theme to urban growth, policy orientated urban
sprawl ambiguity produces difficulties for both local governments and
researchers in monitoring and identifying solutions when the very
concepts aren't defined by metropolitan agencies.\\
Defining urban sprawl as a concept has highlighted ambiguity and
conceptual vagueness (Bennett, 2001; Bhatta, 2010; Bhatta et al., 2010).
Galster et al. (2001) undertook an extensive evaluation of sprawl within
the literature that revealed sprawl can be alternatively or
simultaneously referred to as patterns of land use, processes of land
development, causes of land use behaviours and consequences of land use
behaviours (Bhatta, 2010; Bhatta et al., 2010; Galster et al., 2001).
Furthermore, to add to the complexity, sprawl is also referred to as a
noun and a verb. As a noun it signifies a condition representing an
urban area at a particular time, whereas when used as a verb it defines
a stage or process of development (Bhatta, 2010; Galster et al., 2001).
Whilst a definitive definition of sprawl is contested, a general
academic consensus exists that urban sprawl is characterised by
unplanned and uneven pattern of growth resulting in inefficient resource
utilisation (Bhatta, 2010; Bhatta et al., 2010; Sun et al., 2013), which
is the definition followed in this research.

\subsection{Urban growth and sprawl in Earth
observation}\label{urban-growth-and-sprawl-in-earth-observation}

It is important to establish that every type of urban growth should not
be considered sprawl (Bhatta, 2010). Whilst the notion of sprawl
attracts a negative connotation in terms of environmental and societal
impacts, types of urban growth such as infill development are considered
resolutions to urban sprawl. Consequently, sprawl should not be used
interchangeably with urban growth and must be considered as its own
separate entity. However, within EO studies this is often unfeasible due
to similarities of materials and geometries that compose both urban
sprawl and urban growth being indiscernible using medium resolution EO
imagery. Therefore the majority of studies focus on monitoring urban
growth as a holistic concept (which could encompass sprawl) as being the
change in developed land between two or more time periods (e.g.~Luo et
al., 2014; Schneider and Mertes, 2014; Sexton et al., 2013; Song et al.,
2016).

\section{The urban heat island
effect}\label{the-urban-heat-island-effect}

\subsection{Introduction}\label{introduction-1}

One of the main environmental impacts of urbanisation is the UHI effect,
whereby urban areas obtain comparatively higher atmospheric and surface
temperatures than surrounding rural areas (Cai et al., 2016; Howard,
1988; Hu and Brunsell, 2015; Voogt and Oke, 2003; Xie and Zhou, 2015).
The modification of land cover properties as a result of urbanisation
has been identified as the most extreme cumulative effect of land cover
change, permanently influencing atmospheric energy exchange and altering
local and regional climate change (Cai et al., 2016; Howard, 1988; Hu et
al., 2015; Voogt and Oke, 2003; Xie and Zhou, 2015). Thus, UHIs are
considered one of the major problems posed to humans in the 21st
century, with the importance of understanding heat risk highlighted by
extreme temperatures overtaking flooding and rising to the third highest
cause of global disaster mortality (11.3\%) between 2006 and 2015,
behind only earthquakes (51.2\%) and storms (24.9\%) (Centre for
Research on the Epidemiology of Disaster and the United Nations Office
for Disaster Risk Reduction, 2016).

\subsection{Urban heat island theory}\label{urban-heat-island-theory}

The UHI phenomena is created through a shift in energy balance toward
sensible heat over latent heat. The former represents heat exchanged by
a body that changes temperature (e.g.~conduction, convection or
advection) whilst the latter is defined as energy released or absorbed
by a body (e.g.~evapotranspiration) (Sexton et al., 2013; Wong et al.,
2013). The creation of UHIs is the result of two major factors: an
increased spatial coverage of dark surfaces and the absence or removal
of vegetation (Frumkin, 2002; Voogt, 2004). Dark urban surfaces obtain a
low albedo (albedo being the fraction of shortwave radiation reflected
from an object, in comparison to natural surfaces). Consequently,
impervious surfaces (such as asphalt and concrete) absorb and retain a
large proportion of the shortwave radiation during the daytime,
reemitting this energy as long wave energy (or sensible heat) during the
day (based on surface properties) and particularly at night (Zhao et
al., 2016). However, due to the three dimensional structural arrangement
of modern cities, a low Sky View Factor (SVF), defined as the ratio of
radiation received (or emitted) by a planar surface to the radiation
emitted (or received) by the entire hemispheric environment, frequently
precludes efficient heat dispersion (Abutaleb et al., 2015; Frumkin,
2002; Kalnay and Cai, 2003; Lindberg and Grimmond, 2010; Sharifi and
Lehmann, 2014; Voogt, 2004; Watson and Johnson, 1987; Zhao et al.,
2016). The absence or removal of vegetation reduces solar radiation
blocking to urban surfaces which inhibits heat emittance, diminishing
evapotranspiration that cools the atmosphere through ambient heat
dissipation (Frumkin, 2002). Further factors including air speed, cloud
cover, cyclic solar radiation, building material type and anthropogenic
energy sources can exacerbate this phenomena in a temporary
(e.g.~weather patterns) or more permanent (e.g.~built environment)
manner (Rizwan et al., 2008; Sheng et al., 2015).

Contributing UHI factors are commonly described through the energy
balance equation whereby net radiation, defined as the balance between
incoming and outgoing energy at the top of the atmosphere equates to the
sum of sensible heat (energy heating the air), latent heat (energy used
for evaporation) and surface conductive heat (energy heating the ground)
fluxes. When applying the energy balance equation to UHI concepts
sensible heat is representative of building thermal properties, latent
heat is a function of vegetation coverage and surface conductive heat is
dependent on surface albedo. Recent iterations have also included
anthropogenic heat sources, representative of systems that emit heat
such as air-conditioning and automobiles combined with net radiation
equating to the sum of surface, latent and conductive heat fluxes
(Arnfield, 2003).

\subsection{Urban heat island
categories}\label{urban-heat-island-categories}

This section describes UHI categories and temporal forms, enabling
appropriate terminology inclusion when discussing monitoring approaches
leading to identification of current research and policy issues in
relation to global and metropolitan sustainable city development goals,
the latter part of the overall thesis aim.

\subsection{Heat island layers}\label{heat-island-layers}

The UHI concept can be decomposed into three subcategories: the Boundary
Layer Heat Island (BLHI), the Canopy Layer Heat Island (CLHI) and the
Surface Heat Island (SHI) (Figure 1 4). BLHI and CLHI refer to warming
of the urban atmosphere, with SHI indicating urban surface heat (Fabrizi
et al., 2010; Voogt, 2004). The UCL and SHI are both located within the
Urban Boundary Layer (UCL), extending up to around 1 km or more above
the surface during daytime and shrinking to around hundreds of meters or
less at night. The UCL is the layer of air closet to city surfaces and
extending upwards to average tree or building height, whilst the SHI
depicts the temperature Earth's surface as seen from above (Schwarz et
al., 2012).

Figure 1 4. The main Urban Heat Island (UHI) components of the urban
atmosphere. Adapted from Voogt (2004) and the USA Environmental
Protection Agency (2008).

\subsection{Heat island temporal form}\label{heat-island-temporal-form}

Two different temporal forms of UHI exist: daytime and nighttime.
Daytime UHI is largely driven by solar radiation and the thermal
properties of urban surface materials, compared to nighttime UHI which
is predominately controlled by the release of solar energy trapped
during the day and additional anthropogenic energy sources (Zhao et al.,
2016). Consequently, the temporal alteration of heat source generates
two unique urban heat island profiles, daytime and nighttime. No clear
consensus exists within the literature when identifying whether surface
UHIs of the day (Cheval and Dumitrescu, 2014; Sun et al., 2015; Tran et
al., 2006; USA Environment Protection Agency, 2008; Weng et al., 2004)
or night (Fabrizi et al., 2010; Kenward et al., 2014; Sheng et al.,
2015; Zhou et al., 2015) are more intense, with differences being based
upon geographic location.

\section{Urban heat island impacts}\label{urban-heat-island-impacts}

Elevated temperatures resulting from UHIs have been associated with
detrimental health consequences (Goggins et al., 2012; Michelozzi et
al., 2009; Tan et al., 2010), increased energy requirements (Santamouris
et al., 2015), heightened emissions and economic expenditure (AECOM
Australia, 2012; Frumkin, 2002; USA Environment Protection Agency,
2008). The following sections outline the effects of the UHI in terms of
social, environmental and economic impacts.

\subsection{Social impacts}\label{social-impacts}

The direct effect of heat on populations was first explored by Buechley
et al. (1972), who established an exponential increase in the mortality
rate relative to maximum temperature (Tan et al., 2010). Typically skin
receives 5-10\% of inactive cardiac output, during heat stress this can
rise to between 50 and 70\%. Consequently in order to maintain healthy
blood pressure cardiac yield must be increased generating additional
tension on the heart (Wong et al., 2013). Heat-related illnesses ensue
if heat gain cannot be dissipated through physiological or
thermoregulatory processes (Loughnan et al., 2013). Thus health issues
can range from mild to life-threatening and commonly pertain to the
cardiovascular and respiratory systems, including heat syncope or
fainting, heat edema or swelling, heat tetany or hyperventilation (Basu
and Samet, 2002; Frumkin, 2002; Loughnan et al., 2013). Excess heart
strain can further exacerbate underlying conditions (e.g.~ischemic heart
disease and respiratory illnesses) or the health of vulnerable groups
(e.g.~elderly, high population density residents or high rise living
residents) (Buchin et al., 2015; Tomlinson et al., 2011). The impact of
the UHI effect on health is frequently presented in relation to
respiratory hospital admissions and changes in mortality (Lowe, 2016;
Michelozzi et al., 2009). For example, population adjusted excess
mortality rates during the 1998 Shanghai heatwave were estimated at 27.3
per 100,000 within the urban area compared to only 7 per 100,000 in the
exurban districts (Tan et al., 2010). In Hong Kong a 1 °C rise in
temperature above 29 °C was associated with a 4.1\% increase in
mortality in areas with a high UHI intensity, compared to only a 0.7\%
in areas with a low UHI intensity (Goggins et al., 2012). Whilst in six
``Mediterranean'' (Barcelona, Ljubljana, Milan, Rome, Turin, and
Valencia) and six ``North-Continental'' (Budapest, Dublin, London,
Paris, Stockholm, and Zurich) cities a 2.1\% and 1.2\% increase in
respiratory admissions was respectively observed across all age groups,
increasing to 4.5\% and 3.1\% in the 75 plus age category (Michelozzi et
al., 2009).

\subsection{Environmental impacts}\label{environmental-impacts}

In terms of energy usage, for each degree of ambient temperature rise
the increase in peak electricity load has been estimated between 0.45
and 4.6\%, corresponding to around 21 W per degree rise per person
(Santamouris et al., 2015). This is particularly problematic in
countries where the majority of energy originates from fossil fuel
combustion (e.g.~Australia and USA). The increased energy requirement
can elevate air pollutants such as sulphur dioxide (SO2), nitrogen
oxides (SOx), particulate matter (PM), carbon monoxide (CO) and mercury
(Hg) all of which are considered harmful to human health (Frumkin, 2002;
USA Environment Protection Agency, 2008). For example, using data across
25 USA communities, a rise of 10 μg/m3 in two day average PM2.5
(particles with a diameter of less than 2.5 μm) mass concentration was
associated with a 0.74\% increase in non-accidental deaths (Franklin et
al., 2008).

\subsection{Economic impacts}\label{economic-impacts}

The collective impacts of the UHI were quantitatively estimated in the
first ever economic assessment undertaken by the City of Melbourne
(2012), with the annual UHI cost estimated at AUD 300 million. This was
composed of costs associated with factors and services including health,
transport, energy, anti-social behaviour, tree maintenance and animal
care and mortality (AECOM Australia, 2012). Of these categories
increased health costs dominated (AUD 282 million) overall expenditure,
due to the dangers associated with extreme temperatures on human life
and subsequent predicted mortality rise. The established impacts
associated with the UHI highlight the importance of effective mitigation
measures. However, recent research has estimated the impact of the UHI
could increase the percentage of Gross Domestic Product (GDP) lost by
0.71\% (in 2050) and 1.04\% (in 2100) for the low Green House Gas (GHG)
scenario and 0.80\% (in 2050) and 1.79\% (in 2100) under the very high
GHG scenario due to exclusion from Global Climate Change (GCC) scenarios
(Estrada et al., 2017). As a result the UHI effect has been included in
updated international polices such as the Sendai Framework for Disaster
Risk Reduction (UNISDR, 2015), SDGs (Osborn et al., 2015) and CRF (ARUP
and The Rockefeller Foundation, 2015), yet these policies currently lack
any methodological approach for UHI monitoring. Consequently, it is of
vital importance to develop effective data-driven mitigation measures
and planning policies to ensure the future sustainability of our cities.

\chapter{Monitoring urban growth and the urban heat island effect using
Earth
observation}\label{monitoring-urban-growth-and-the-urban-heat-island-effect-using-earth-observation}

\section{Introduction}\label{introduction-2}

Chapter 1 outlined the concepts and issues of poorly planned urban
development and the Urban Heat Island (UHI) effect. Chapter 2 builds
upon this base knowledge through discussing current monitoring
approaches and associated limitations that are addressed within this
thesis.

\section{Monitoring land cover
change}\label{monitoring-land-cover-change}

Land cover data can be collected using traditional field surveys,
however, for expansive urban metropolitan areas these are often
unfeasible due to the required funding, person hours, strategic planning
and annual replication for complete temporal and spatial coverage. As a
result, data derived from field surveys is often incomplete, spatially
aggregated, and temporally and geographically limited. Comparatively,
satellite remote sensing enables efficient extraction of geographic land
cover changes with considerable cost reductions in a timely and synoptic
manner (Powell et al., 2007; Yuan et al., 2005). This is achieved
through classification of unique surface reflectance signatures per
pixel, being the fraction of incoming solar radiation reflected by the
Earth's surface over a defined area (e.g.~30 m2) into predefined land
cover classes.

Classifying an Earth Observation (EO) image in determining land cover
change can be summarised by three main steps: preprocessing,
classification and output evaluation. The majority of recent research
advancements focus on classification, with comparatively fewer and more
established methods for preprocessing and output evaluation. Three main
image analysis approaches exist for extracting land cover estimates from
EO data: classification algorithms, spectral indices and data-fusion
approaches. Classification algorithms assign one or more user-defined
land cover classes to a pixel in the digital image. For example,
GlobeLand30 provided global land cover estimates divided into 10
classes, namely water bodies, wetland, artificial surfaces, cultivated
land, snow/ice, forest, shrubland, grassland, bare earth and tundra from
Landsat data with an overall accuracy exceeding 80\% (Chen et al.,
2015). However, the difficulties of extracting built-up areas is a known
issue within the remote sensing community, resultant from spectral
similarities between natural surface materials such as bare soil and
man-made materials such as impervious surfaces leading to spectral
confusion during classification (Herold et al., 2002; Lu et al., 2011;
Varshney and Rajesh, 2014). New per-pixel spectral indices, such as the
Normalized Difference Built-up Index (NDBI) (Zha et al., 2003), provide
an alternative approach in determining the presence of a sole land cover
class (e.g.~urban), with a user-determined threshold establishing the
value at which a pixel is assigned to the land cover type (Angiuli and
Trianni, 2013; Xu, 2008; Zha et al., 2003). However, Schneider (2012)
suggested analysis must move beyond mere consideration of spectral
information to the temporal, spatial or polarimetric domain in order to
resolve misclassification, particularly in an urban environment. In this
sense, additional variables are obtained or computed and appended to
original imagery for classification algorithm accuracy improvement. In
the following sections an overview of preprocessing and output
evaluation is provided, with a comprehensive literature review of
current classification methodologies.

\subsection{Image preprocessing}\label{image-preprocessing}

Image preprocessing entails correction for noise unattributed to surface
reflectance, such as radiometric (atmospheric), geometric (image
projection) and topographic (physical features) errors (Hansen and
Loveland, 2012). Satellite data results in images experiencing differing
radiometric conditions defined as the sensitivity of the sensor to
incoming reflectance, due to variations in atmospheric conditions, solar
illumination, sensor calibration, view angle and soil and vegetation
changes (Du et al., 2002; Yang and Lo, 2000). Similarly, geometric
misalignment and slope orientation in relation to incoming solar
radiation can result in inconsistences when undertaking scene
classification and thematic evaluation (Richter et al., 2009). The
majority of studies exploring land use and land cover change have
implemented medium spatial resolution (30 m) Landsat imagery, obtaining
the longest, free temporal image repository of consistent medium spatial
resolution data, with a temporal resolution of 16 days (Bagan and
Yamagata, 2012a; Kressler and Steinnocher, 2001; Lu et al., 2011; Sexton
et al., 2013; Sundarakumar et al., 2012; Tan et al., 2009). Landsat data
are distributed as a surface reflectance product achieved through the
Landsat Ecosystem Disturbance Adaptive Processing System (LEDPAS) and
the Landsat 8 Surface Reflectance algorithm (L8SR) otherwise known as
the Landsat 8 Surface Reflectance Code (LaSRC) for correction of
atmospheric conditions (Hansen and Loveland, 2012; USGS, 2015). The
former corrects for atmospheric effects using the Second Simulation of a
Satellite Signal in the Solar Spectrum (6S) radiative transfer model,
whilst the latter implements an internally developed algorithm (Hansen
and Loveland, 2012; USGS, 2015). Additional processing to Level 1
Terrain-corrected data (L1T) corrects for both geometric and topographic
errors using ground control points and a Digital Elevation Model (DEM)
from the Global Land Survey (GLS) 2000 data set (Hansen and Loveland
2012). However, owing to the requirement of several parameters
(i.e.~Aerosol Optical Thickness (AOT), ozone and air temperature) for
surface reflectance derivation assumptions or models of values are often
implemented (Ju et al., 2012). Consequently, for removal of remaining
post-atmospheric correction noise such as the brightening effect of
cloud or darkening of cloud shadow, Sexton et al. (2013) put forward the
notion of image standardisation based on pre-defined band specific
values and subsequent normalisation for reduced inter-annual surface
reflectance values. When classifying EO imagery over multiple years this
approach permits the use of a single classification model as opposed to
individual classification for each time point considered.

\subsection{Image classification}\label{image-classification}

Holistic image classification is achieved through a supervised or
unsupervised methodology. In a supervised classification the user
selects sample pixels termed Regions of Interest (ROI) that are
representative of predefined land cover classes. ROIs are then input to
a classification algorithm that identifies pixels with similar
reflectance values to each provided land cover class for entire image
classification. During an unsupervised classification an algorithm
automatically separates image pixels into clusters obtaining similar
spectral characteristics, with only a user defined number of classes.
Owing to the majority of recent literature focusing on supervised
classifiers due to the complexity and size of datasets this review shall
be limited to supervised methodologies (Schneider, 2012).

\subsection{Recent classification
methodologies}\label{recent-classification-methodologies}

Classifier selection is dependent on the nature of input data and
desired output, defined as parametric or nonparametric. Parametric
algorithms make assumptions regarding ROIs selected for training, such
as a Gaussian distribution, whereas nonparametric algorithms do not make
this assumption (Donnay and Unwin, 2001; Jensen, 2005). Popular
traditional nonparametric classification algorithms include density
slicing, parallepiped, minimum distance to mean, and nearest-neighbour,
with Maximum Likelihood (ML) the most widely utilised parametric
classifier (Jensen, 2005).

\subsubsection{Maximum likelihood}\label{maximum-likelihood}

The ML classification is based on Bayes' Theorem of decision making. It
assigns each pixel to the most probable user-defined land cover class,
rather than the minimum distance, through considering both the variances
and covariances of class signatures (Atkinson and Lewis, 2000; Jensen,
2005). The ML algorithm permits specification of prior classification
probability information (i.e.~expected frequency of classes per scene).
However, in reality, information of this sort is rarely available, with
the majority of ML applications assuming equal class probability per
scene and assigning land cover classification to pixels based upon the
highest probability. The advantage of the ML classifier pertains to
assignment based on probability often used in determining land cover
(Fuller et al., 1994).

\subsubsection{Spectral mixture
analysis}\label{spectral-mixture-analysis}

More recently, parametric Spectral Mixture Analysis (SMA) and
subsequently machine learning or `expert systems' have been implemented
to solve classification problems (Jensen, 2005; Okujeni et al., 2014).
SMA considers the selection of spectrally unique endmembers, with image
data being assigned the most appropriate match (Powell et al., 2007). It
is based on the assumption that reflectance measured at each pixel is
represented by the linear sum of endmembers weighted by the associated
endmember fraction. In standard SMA a set number of representative
endmembers, commonly between two and five, are extracted, with the
entire image being modelled on their spectral characteristics (Powell et
al., 2007). Endmember extraction normally revolves around identification
of spectral extremes (e.g.~Adams, 1995). However, selection of a limited
number of extreme endmembers results in an inability to adequately
represent the high spectral heterogeneity of the urban landscape (Powell
et al., 2007). Consequently endmembers may not fully represent image
spectral variability or a pixel may be modelled by endmembers that do
not represent materials within its field of view. Both factors result in
a reduction of classification accuracy (Powell et al., 2007). Due to
being an original approach that could consider multiple endmembers
whilst improving accuracy of ML and minimum distance approaches SMA has
been used in establishing Vegetation-Impervious surface-Soil (V-I-S)
fractions (Phinn et al., 2002) and analysing impervious surface
distributions (Wu and Murray, 2003).

\subsubsection{Multiple endmember spectral mixture
analysis}\label{multiple-endmember-spectral-mixture-analysis}

Multiple Endmember Spectral Mixture Analysis (MESMA) extends this
methodology through permitting the number and type of endmembers to
alter on a per pixel basis attempting to represent the inherent spectral
variability within land cover types over the entire image as a linear
combination of constituent components (Okujeni et al., 2015, 2013;
Powell et al., 2007; Weng and Pu, 2013). Mixture models are iteratively
calculated for each pixel, comparing all possible endmember
formulations, deriving the fit between measured and modelled signals.
The model obtaining the lowest Root Mean Square Error (RMSE) is
designated to the pixel (Okujeni et al., 2013). For each land cover
class the MESMA library should obtain sufficient spectra to competently
represent spectral variability. However, as the overall number of
endmembers increases computational efficiency exponentially decreases.
Thus, the endmember library should remain adequately small to maximise
computational efficiency, whilst obtaining land cover spectral diversity
within selected spectra (Powell et al., 2007). Due to the advantages of
multiple endmember selection MESMA has been implemented in classifying
land cover (Franke et al., 2009) and mapping forest fire burn severity
levels (Quintano et al., 2013).

\subsubsection{Machine learning
algorithms}\label{machine-learning-algorithms}

In contrast, Machine Learning Algorithms (MLAs) use an automated
inductive approach for identification of patterns in data (Cracknell and
Reading, 2014). The majority of research focusing on MLAs surrounds the
predication of land cover from multi-spectral or hyperspectral surface
reflectance measurements (Angiuli and Trianni, 2013; Braun et al., 2012;
Rodriguez-Galiano et al., 2012; Schneider, 2012). MLA classification is
derived by a discrimination function y=f(x), with inputs expressed as
vectors d of the form (x\_1,x\_2,\ldots{},x\_d), where y is a definitive
set of c class labels (y\_1,y\_2,\ldots{},y\_c). Using instances of x
and y supervised machine learning trains the classification model,
mapping image data to defined classes (Cracknell and Reading, 2014).
Popular MLAs include Artificial Neural Networks (ANNs), Random Forests
(RFs) and Support Vector Machines (SVMs).

\subsubsection{Artificial neural
network}\label{artificial-neural-network}

Nonparametric ANNs are an interconnected group of nodes that use
mathematical methods to process information in a self-adaptive system,
attempting to `mimic' a human brain (Bhatta, 2010; Hu and Weng, 2009).
The Multi-Layer Perceptron (MLP) feed forward network is the most
popular ANN; obtaining three layers - one input, one hidden and one
output layer - each comprising of several nodes (artificial neurons).
The input layer represents the original image, with each band
representing one node. Classification is undertaken in the hidden layer,
with results presented in the output layer. The learning ability
originates from the learning algorithm, with the most popular being
Back-Propagation (BP) otherwise known as delta rules (Hu and Weng,
2009). Learning is achieved through node weight assignment, with
training samples input into the model. If the difference between the
produced results and test sample is larger than the initial threshold,
weights are altered for difference minimisation. The process is iterated
until a pre-defined accuracy level is obtained or maximum iterations
reached and a classified land cover output is produced (Candade and
Dixon, 2004; Cracknell and Reading, 2014; Hu and Weng, 2009). ANNs have
been widely used due to their robustness and ability to learn complex
patterns, successfully implemented in ship detection (Tang et al.,
2015), tree detection (Malek et al., 2014) and land use classification
(Cheng and Han, 2016; Hu and Weng, 2009; Pacifici et al., 2009).

\subsubsection{Random forest}\label{random-forest}

RFs are a nonparametric ensemble learning method that implement a
majority vote system to predict classes based on data partition from
multiple Decision Trees (DT) (Breiman, 2001; Cracknell and Reading,
2014). Multiple trees are created, using a random subset of input
features to reduce generalisation error, with the end user specifying
the number of trees to be developed and number of features at each node.
Each tree implements a bagging sample permitting growth based on
differing training subsets, with a search across a random selection of
input variables for derivation of a split per node (Cracknell and
Reading, 2014; Gislason et al., 2006; Rodriguez-Galiano et al., 2012).
Bagging facilitates training data creation through randomly resampling
the original dataset, with each selected subset for tree growth
containing a proportion of the training dataset. Samples not selected
are input to the Out Of Bag subset (OOB). OOB samples not utilised for
tree training can be classified by the tree for performance evaluation
(Rodriguez-Galiano et al., 2012). DT design requires the determination
of an attribute section and pruning method. The random forest classifier
implements the Gini Index as the attribute selector method, measuring
the impurity of an attribute compared to classes (Pal, 2005). DT can be
constrained through a termination criterion threshold limiting growth
size and therefore overfitting, termed pre-pruning. Additionally
post-pruning techniques permit overall performance evaluation, due to
being pruned with validation data. However, Breiman (1999) suggested
that whilst the number of trees increases the generalisation error
always converges without overfitting due to the Strong Law of Large
Numbers, which states that the average results obtained from a large
number of trials (or trees) should be near the expected value and will
become closer with the more trials performed (Rodriguez-Galiano et al.,
2012). Thus, for classification, each tree within the RF inputs a vote
for the most popular class, with the output classification determined by
the majority of tree votes (Gislason et al., 2006). RF are advantageous
over other ensemble classification methodologies such as boosting and
bagging through an improved methodological process and less intensive
computational requirement being used in instances to classify: land
cover (Gislason et al., 2006) and tree species (Immitzer et al., 2012).

\subsubsection{Support vector machine}\label{support-vector-machine}

The nonparametric SVM classifier identifies an optimal maximum margin
separating hyperplane, dividing the dataset into the predefined number
of classes, with points on the margins termed support vectors (Braun et
al., 2012; Foody and Mathur, 2006, 2004; Mountrakis et al., 2011; Qian
et al., 2014; Vapnik and Chervonenkis, 1971). The underlying benefit of
SVM is known as structural risk minimisation, whereby SVMs are able to
minimise error on unseen data without prior assumptions made on the data
probability distribution (Mountrakis et al., 2011). SVMs are linear
binary classifiers assigning participant pixels into one of two
possibilities. However, remote sensing derived land covers are often not
linearly separable due to cluster overlap. Consequently implementation
of soft margin and kernel methods aid inseparability through
transforming data into high dimensional feature spaces (Euclidean or
Hilbert) utilising non-linear functions to identify linear solutions
(Braun et al., 2012; Mountrakis et al., 2011). In order to prevent over
fitting SVM implements a two-dimensional grid search using stratified
crossvalidation to search for the kernel (g) and regularisation
parameter (C); (g) defines the width of the Gaussian kernel function
whilst (C) controls training data and decision boundary maximisation
plus margin errors (Zhu and Hastie, 2005). For derivation of more than
two land cover classes additional methodological processes are required,
common methods include one-against-all, one-against-one and directed
acyclic graph SVM, whereby the binary nature of SVM is iterated in
differing formants to derive the appropriate land cover classification
(Chih-Wei et al., 2008; Mountrakis et al., 2011). SVMs are one of the
most prominent and effective MLA due to structural risk minimisation
applied to a variety of applications including land cover change
detection (De Morsier et al., 2013), airport detection (Tao et al.,
2011) and road extraction (Cheng and Han, 2016; Das et al., 2011).

\subsection{Comparison of recent classification
methodologies}\label{comparison-of-recent-classification-methodologies}

Image classification accuracy is dependent on the selected
classification methodology and the choice of internal parameters (Huang
et al., 2002; Watanachaturaporn et al., 2008). Due to the parametric
nature of the ML classifier it can often fail to represent land cover
that might be multimodal, thus in certain circumstances ML has been
outperformed by alternative classification algorithms (Melgani and
Bruzzone, 2004; Mountrakis et al., 2011; Otukei and Blaschke, 2010;
Watanachaturaporn et al., 2008). Similarly MESMA classification can be
inefficient owing to additional computational demands associated with an
increasing numbers of endmembers which often precludes selection and is
consequently considered a more traditional method when compared to MLAs
(Okujeni et al., 2015; Ram and Wang, 2013). In a comparison of ML, DT
(e.g.~RF), ANN and SVM classifiers, Watanachaturaporn et al. (2008) and
Kotsiantis et al. (2006) found the SVM classifier to produce optimal
accuracy. Similarly Huang et al. (2002) found SVMs obtain a higher
accuracy than ML, DTs and ANNs indicating that the superior performance
of SVM is attributed to the derivation of an optimal separating
hyperplane (Foody and Mathur, 2006, 2004; Huang et al., 2002; Mountrakis
et al., 2011). Whilst no single MLA can uniformly outperform all other
MLAs across all data sets, in terms of overall accuracy the majority of
literature preferences implementation of SVM due to its
self-adaptability, efficient learning speed and limited training data
requirements (Kotsiantis et al., 2006; Mountrakis et al., 2011).

\subsection{Image spectral
combinations}\label{image-spectral-combinations}

The reliable and accurate identification and extraction of built-up
areas from medium resolution EO imagery (e.g.~Landsat) is a known issue
within the remote sensing community; originating from spectral
heterogeneity of urban surfaces often resulting in spectral confusion
during image classification (Herold et al., 2002; Lu et al., 2011;
Varshney and Rajesh, 2014). Consequently new spectral indices, which in
the most part do not require classification have been postulated as an
alternative and more computational efficient approach.

Zha et al. (2003) proposed a built up index termed the Normalized
Difference Built-up Index (NDBI) algorithm for identification of built
up regions using the reflective bands: Red, Near-Infrared (NIR) and
Mid-Infrared (MIR). NDBI makes the assumption that built up area has a
high spectral reflectance in the MIR compared to the NIR. However, MIR
vegetated spectral response can increase above NIR under drier
conditions (Gao, 1996; Xu, 2008). Thus, Zha et al. (2003) implemented
the Normalised Difference Vegetation Index (NDVI) to filter noise
arising from vegetation. Nevertheless Xu (2008) stated that sole use of
original spectral bands for construction of a built-up land index is
inappropriate due to the composition of complex spectral features.

Consequently Xu (2008) followed the methodological framework of Ridd
(1995), that the spatial composition of urban areas can be decomposed
into Vegetation-Impervious surface-Soil creating the V-I-S model with
the inclusion of water, grouping the urban area into: built up land,
vegetation and open water (Ridd, 1995). Three indices of NDBI, the Soil
Adjusted Vegetation Index (SAVI) and Modified Normalized Difference
Water Index (MNDWI) represented the land cover categories respectively.
MNDWI modifies the Normalized Difference Water Index (NDWI) through
selection of the MIR band in place of the NIR band, remediating built up
land noise for open water selection. SAVI was preferenced over NDVI due
to greater sensitivity in detecting vegetation in low-plant covered
regions such as urban areas, estimated to work with plant cover as low
as 15\% compared to NDVI at 30\% (Xu, 2008). Aforementioned indices
extracting unique features were then combined in the Index-based
Built-up Index (IBI). However, the intrinsic issue of IBI pertains to
selection of an appropriate user defined correction value for SAVI
ranging from 0 for high plant densities to 1 for low plant densities.
Furthermore complete land cover is assumed to be adequately modelled
from built land, vegetation and water. Thus, analysis has resulted in
urban area remaining mixed with bare earth, requiring additional polygon
layers defining the urban region from an unspecified source to filter
erroneous built up land for definitive extraction (Stathakis et al.,
2012; Sun et al., 2015; Xu, 2008; Zha et al., 2003). Additionally, the
nature of determining appropriate singular threshold values over
heterogeneous urbans fails in global practicality and has the potential
for the introduction of localised errors impacting reliability (Xu,
2008).

In contrast to the threshold approach presented by Xu (2008), Angiuli
and Trianni (2013) proposed the Normalised Difference Spectral Vector
(NDSV). Due to multiple indices presented throughout literature NDSV
attempts a simultaneous merge for the production of intrinsically
normalised globally consistent data whilst reducing ambiguities
associated within individual indexes (Angiuli and Trianni, 2013; Patel
et al., 2015). NDSV computes all possible indices through the
combination of all bands. Thus, with 6 Landsat bands a total of 30
indexes are generated, but due to the symmetry of definition, 15 are
negative representations of other indexes (Angiuli and Trianni, 2013;
Patel et al., 2015). NDSV creates a normalised signature per pixel and
is subsequently classified. Nevertheless, any index is founded upon
assumptions, for example, NDVI is based on the rationale that green
plants absorb solar radiation in the photosynthetically active radiation
spectral region (400 -- 700 nm) and reflect radiation in the NIR region.
Therefore NDVI is usually highly correlated with Leaf Area Index (LAI)
and has found to be sensitive to canopy background variations such as
soil visible through the canopy (Jensen, 2009). Consequently NDVI can be
unsatisfactory, especially when mapping senesced vegetation owing to
reduced absorption in the visible bands and reflection in the NIR band
(Jensen, 2005). This methodology also fails to directly extract urban
extents owing to the requirement of a classification model.

\subsection{Data fusion methodologies}\label{data-fusion-methodologies}

Recent data fusion methodologies combine additional or computed data to
existing spectral bands in order to improve classification accuracy
(Rodriguez-Galiano et al., 2012). Popular approaches can be categorised
into spatial, temporal and polarimetric domains extracting additional
information from texture, temporal composites and radar respectively.
The following sections outline and compare these procedures, determining
current research trends and establishing the most appropriate
methodological approach in quantifying the temporal urban growth section
of the overall thesis aim.

\subsubsection{Spatial domain}\label{spatial-domain}

Texture analysis provides a representation of the visual characteristics
of an image permitting incorporation of spatial information into image
(e.g.~Landsat) classification found to produce more accurate
classifications of heterogeneous land covers such as urban
(Møller-Jensen et al., 2005; Rodriguez-Galiano et al., 2012; Zhou and
Troy, 2008). Co-occurrence texture measures such as mean, variance,
homogeneity, contrast, dissimilarity, entropy, second moment and
correlation are computed using a moving rectangular window surrounding a
central pixel. Nevertheless, an overarching issue pertains to window
size; it must be large enough to capture variance, yet small enough to
represent homogenous land cover (Møller-Jensen et al., 2005).
Consequently, the window-based approach tends to smooth boundaries
between discrete land cover types determined from medium-coarse
resolution imagery, with the appropriate window size being difficult to
discern and a rectangular window not necessarily representative of real
land coverage (Møller-Jensen et al., 2005). Very high resolution
(\textless{} 1 m) surface reflectance imagery, Light Detection and
Ranging (LiDAR) and stereo imagery such as that procured during Perth's
Urban Monitoring project and the State of Indiana's strategic plan can
overcome these limitations, but are associated with high financial
outlay and often infrequent repeat collections that currently preclude
extensive temporal monitoring (e.g.~15 years) (Caccetta et al., 2012;
The State of Indiana, 2017).

\subsubsection{Temporal domain}\label{temporal-domain}

Annual and multi-seasonal temporal image composites increase class
spectral separability through stacking imagery from multiple dates into
a single image (Bhatta, 2010; Castrence et al., 2014; Schneider, 2012;
Sexton et al., 2013; Yuan et al., 2005; Zhu et al., 2012). This follows
the logic that natural surfaces obtain a type of cyclical pattern
resulting from changes in the proportion of land cover (e.g.~mixtures of
vegetation, soil and water) based on the time of year (Jensen, 2005).
However, when natural surfaces are replaced with impervious structures
the fluctuation will cease owing to the conversion to built-up land
cover generally being unidirectional, identifiable from a multi-temporal
signature in spectral space (Castrence et al., 2014). Regardless, the
premise of this method is founded upon the assumption that limited or no
change will have occurred within a complete temporal period of stacked
imagery. Ideally, each variable used in classification should enable
additional refinement for improved accuracy. Nevertheless, due to the
number of bands within the multi-temporal image composite high variable
correlation may be prevalent (Bhatta, 2010; Zhu et al., 2012).
Redundancy can be overcome through principal component transformation,
with components containing significant variance selected for
classification (Bhatta, 2010). Whilst Zhu et al. (2012) acknowledged
this issue through investigating the effect of increasing variables
during classification they concluded that although some variables
contribute relatively little, the trend is straightforward; more
independent data results in higher classification accuracy.

\subsubsection{Polarimetric domain}\label{polarimetric-domain}

Synthetic Aperture Radar (SAR) data are playing an increasingly
important role in remote sensing owing to all weather operational
ability (Zhu et al., 2012). Although SAR images over urban areas provide
low quality images due to problems associated with radar imaging in such
an environment (i.e.~multiple bouncing, layover and shadowing), SAR
texture measures can provide valuable information in discerning urban
areas (Dell'Acqua et al., 2003; Zhu et al., 2012). Isolated scattering
of residential areas and crowded backscatters of inner city high density
areas permit classification refinement, thus textural measures such as
those descried within the spatial domain can aid identification of
alternative urban forms (Zhu et al., 2012). However, the lack of freely
available SAR data that temporally coincides with other satellite
imagery (e.g.~Landsat) frequently precludes extensive use.

\subsection{Output evaluation}\label{output-evaluation}

Accuracy assessment of classified data is key to ensure effective and
appropriate data usage. Accuracy can be determined through visual
inspection, non-site specific analysis, difference imaging, error
budgeting and quantitative assessments (Congalton, 2001). Visual
inspection is often the first step of assessment in ensuring the
production of a valid output, but does not provide numerical
quantification. Non-site specific analysis and difference imaging
compare classified output between an alternative data source for a small
spatial area and complete image respectively, providing a spatial
component to map error. However, these methods fail in determining the
accuracy of each individual land cover class, presented as difference in
area estimates and difference images. Error budgeting estimates the
total error of a project workflow based on analyst attributed values,
combined in an error index (Congalton, 2001). Whilst this assists in
determining and assessing data input, user and methodological error
potential it fails in end user classification output accuracy
estimation. A quantitative accuracy assessment is imperative in order to
accurately report any modelled urban growth estimates, often omitted
from values provided in metropolitan planning documents. An error matrix
is the most common quantitative evaluation of classified remotely sensed
data (Foody and Mathur, 2004; Friedl et al., 2010; Van de Voorde et al.,
2011; Watanachaturaporn et al., 2008). An error matrix is a square array
comparing the number of sample units correctly determined by the
classifier in relation to a data source (e.g.~original image or Google
Earth) per land cover category. Outputs include (i) user's accuracy
defined as the fraction of correctly classified pixels relative to all
others classified as a particular land cover, (ii) producer's accuracy
defined as the fraction of correctly classified pixels compared to
ground truth data, and (iii) overall accuracy that represents the
combined fraction of correctly classified pixels across all land cover
types (Congalton, 2001). Quantitative accuracy metrics of this sort
permit appropriate use of land cover products and parameterisation of
further analysis that expands upon the classified output such as recent
Urban Heat Island (UHI) studies that combine land cover data and
satellite derived temperature.

\subsection{Classification methodological
conclusion}\label{classification-methodological-conclusion}

Mapping urban areas remains a complex challenge owing to the complex
variation of materials and geometries that compose the urban environment
and contribute to mixed spectral signatures (Schneider, 2012).
Methodologies employed for extraction of urban areas from satellite
imagery are diverse and often location dependent, with no current
standardised best practice for urban monitoring. Throughout the
literature spectral, spatial, temporal and polarimetric data have been
used in differing formulations for urban area extraction.

Due to the limited past record of complete SAR data, required temporal
analysis observed within academic and metropolitan studies (e.g.~15
years) is often unfeasible. Additionally, spectral analysis can be
seasonally dependent, whilst spatial analysis can remove underlying
trends through data smoothing and poor representation of real land cover
due to a definitive rectangular moving window. Generation of unique
multi-temporal spectral signatures increases the amount of independent
data available for classification but the underlying assumption of
minimal change between composited images is made. Due to these
limitations, EO data is typically classified as standalone data
(e.g.~surface reflectance) by classification algorithms discussed in
section 2.2.3. However, these methodologies have been found to
significantly over or underestimate urban area by between 50-60\% in
complex landscapes such as the urban-rural frontier (Lu et al., 2011; Wu
and Murray, 2003). Improving our ability to map urban area is currently
an essential challenge due to the potential for classified land cover
products to inform decision making such as determining future
development strategies and informing further environmental analysis and
policies (Bagan and Yamagata, 2014; Hepinstall-Cymerman et al., 2013;
Miller and Small, 2003; Schneider et al., 2005).

\section{Monitoring urban heat
islands}\label{monitoring-urban-heat-islands}

Two forms of temperature affecting urban areas are frequently monitored
in relation to an expanding urban area; air temperature and Land Surface
Temperature (LST). The former often pertains to traditional
meterological monitoring, whereas the latter is based on thermal
measurements made from EO data. The following sections describe these
two methodological approaches.

\subsection{Traditional urban heat island
methodologies}\label{traditional-urban-heat-island-methodologies}

Air temperature represents Urban Canopy Layer (UCL) temperature;
directly impacting human comfort and public health, monitored from
static weather stations (Guo et al., 2014). For example, Shanghai's
heatwave excess mortality rates (Tan et al., 2010), Hong Kong's
UHI-mortality association (Goggins et al., 2012) and Melbourne's UHI
economic assessment (AECOM Australia, 2012) used differenced temperature
data from meteorological stations in rural and urban geographical
locations to determine the UHI Intensity (UHII) (Tan et al., 2010). To
account for the spatial variation in the UHI effect, the temperature
measurements from weather stations are often spatially interpolated. The
accuracy of the interpolated dataset can be dependent on the type of
interpolation method, the number of points available and distance
between them (Hattis et al., 2012). Consequently, whilst studies using
point-based meteorological data provide a broad city scale view of the
UHI, they are impractical for targeted mitigative planning actions
(e.g.~urban greening) due to the limited number of meteorological
stations in many areas.

\subsection{Remotely sensed land surface
temperature}\label{remotely-sensed-land-surface-temperature}

EO data overcomes the limitations of point based methods through
providing near global coverage of LST on a per pixel basis using
instruments that measure in thermal spectral wavebands. LST measurements
are widely used to quantify the impact of land cover type on the Surface
Heat Island (SHI), often related to air temperature in the same
location, resulting in the term Surface Urban Heat Island (SUHI)
(Schwarz et al., 2012; Voogt and Oke, 2003). In the context of
characterising the UHI, LST is typically used, as opposed to air
temperature due to the additional parameters required to compute air
temperature such as surface properties, atmospheric conditions and solar
angles that must be incorporated, assuming data availability during
satellite overpass. Due to their advantages in monitoring temperature,
satellite instruments including the Moderate Resolution Imaging
Spectroradiometer (MODIS) (Wang et al., 2015), Landsat Thematic Mapper
(TM) (Jimenez-Munoz et al., 2014; Rinner and Hussain, 2011; Sobrino et
al., 2004), the geostationary Spinning Enhanced Visible and Infrared
Imager (SEVIRI) (Blasi et al., 2016) and the Advanced Along-Track
Scanning Radiometer (AATSR) (Fabrizi et al., 2010) have been used to
monitor LST and the UHI effect. Nevertheless, current methodologies
frequently fail in planning practicality due to the static temporal
nature through consideration of limited (two or less) EO temperature
images (Li et al., 2011; Tomlinson et al., 2011) alongside aggregation
to broad land cover types or use of singular metrics such as the UHII
(Cao et al., 2010; Imhoff et al., 2010; Zhou et al., 2016). For example
Li et al. (2011) used temperature extracted from two Landsat images
captured in March and July 2001 to infer the effects of landscape
composition and configuration on the UHI in Shanghai. Whilst their
results produced strong correlations between LST and landscape metrics,
selection of single images obtained during spring (March) and summer
(July) disregard the temporal component of LST (e.g.~the complete annual
temperature cycle) and fail to account for potential abnormalities in
temperature on selected days (e.g.~heatwaves) (Li et al., 2011).

Similarly, whilst Zhou et al. (2016) explored the spatio-temporal trends
of the UHI throughout China using daily MODIS LST between 2003 and 2016,
their analysis was restricted to comparison of the UHII using land cover
data from only 2005 and 2010. The use of two classified land cover
images restricted UHI analysis through the assumption of unchanged urban
area between 2003-2007 (for the 2005 image) and 2008-2012 (for the 2010
image). Land cover changes within these timeframes had the potential to
produce erroneous results alongside sole output of the UHII that
precludes quantification of changes in land cover associated with
temperature for targeted policy remediation (Zhou et al., 2016).
Consequently, research must adapt to consider the needs and requirements
of metropolitan development frameworks in order to assist in more
sustainable future metropolitan development.

\subsection{Localised temperature
mitigation}\label{localised-temperature-mitigation}

In response to the UHI effect and updated international policies
outlined in section 1.5 a variety of localised mitigation measures have
ensued, categorised into voluntary and policy themes. The former
represents demonstrative projects and incentives such as Sacramento's
Tree Foundation providing free shade trees to Sacramento residents (USA
Environmental Protection Agency, 2013). The latter incorporates the UHI
into metropolitan frameworks such as Perth and Peel \citet{3.5million}
(Western Australian Planning Commission, 2015a), The London Plan (Mayor
of London, 2016a) and Johannesburg's Spatial Development Framework 2040
(City of Johannesburg Metropolitan Municipality, 2016). \textbf{However,
these policies frequently fail in planning practicality through lacking
any specific methodological requirement}. Consequently local governments
have incorporated quantifiable policy requirements such as Seattle's
Green Factor specifying minimum vegetation requirements, yet lacking
placement guidelines that could result in sub-optimal locations (USA
Environmental Protection Agency, 2013). Other local governments such as
the City of Perth and Fremantle have initiated EO informed Urban Forest
programmes to maintain and increase vegetation coverage (City of
Fremantle, 2017; City of Perth, 2006). However, due to the lack of
scientifically applied UHI mitigation studies and devolution of targets
to local governments, varied, inconsistent and aggregated block scale
LST methodologies provide the potential to misinform vegetation
placement (City of Fremantle, 2017; City of Perth, 2006). The majority
of academic literature implementing remotely sensed data analysing the
UHI effect uses medium-low resolution satellite imagery (e.g.~MODIS, 1
km and Landsat, 30 m), inappropriate for very small scale, localised UHI
mitigation decisions. It is therefore imperative to provide
policy-relatable methodologies in order to facilitate
scientifically-valid decision making in ensuring the sustainability of
our cities.

\chapter{Research aim and objectives}\label{research-aim-and-objectives}

This short chapter outlines the overall research aim and objectives of
this thesis. The aim of this research is to demonstrate the application
of Earth Observation (EO) data in quantifying urban growth and its
impact on the Urban Heat Island (UHI) in order to illustrate its
potential for informing both global and metropolitan sustainable city
development goals. In achieving this aim the research objectives are
extracted from current themes and gaps, presented both in the main
thesis introduction and in each paper style chapter. This is divided
across the four paper style chapters each addressing their own
objectives:

\begin{enumerate}
\def\labelenumi{\arabic{enumi}.}
\item
  Provide a remotely sensed spatio-temporal assessment (paper 1a,
  chapter 4) and associated methodology (paper 1b, chapter 4) of change
  in urban area across the Perth Metropolitan Region (PMR) using a
  consistent methodology.
\item
  Develop an approach to remediate frequent over (or under) estimation
  of urban area classified from medium resolution satellite imagery
  through comparison to very high resolution aerial imagery --- paper 2,
  chapter 5.
\item
  Investigate the spatio-temporal UHI characteristics at the
  sub-metropolitan level using a per pixel approach through (a)
  determining the complexities of the UHI effect and (b) deriving
  associations between land cover and temperature change at the
  intra-urban scale --- paper 3, chapter 6.
\item
  Establish an operational methodology for evidence-based urban planning
  to optimise localised UHI mitigation through the use of EO data and
  open source spatial temperature models --- paper 4, chapter 7.
\end{enumerate}

\section{Study site}\label{study-site}

This section provides an introduction and overview to the PMR in Western
Australia (WA), the study site used throughout this thesis in achieving
the objectives and overall aim. The City of Perth is the State Capital
of WA and has undergone dramatic urban and population growth accredited
to Australia's natural resources boom commencing around the start of the
21st century. Mining and petroleum exports dominate WA's export products
attributing 95\% of export earnings between 2010 and 2011, with sales
rising from AUD 4.7 billion in 1996 to a peak of AUD 121.6 billion
mid-2013 (Department of Mines and Petroleum, 2015). The majority of the
urban growth has been identified as sprawling, outward and low-density
by the Western Australian Planning Commission (WAPC) (Western Australian
Planning Commission, 2015a). This is representative of the `Australian
dream' comprising of detached living within a green suburb on greenfield
urban-fringe sites (Dhakal, 2014; Western Australian Planning
Commission, 2015a). Consequently in comparison to other Australian state
capitals Perth was Australia's fastest growing city (in terms of
population) between 2007 and 2014 whilst only obtaining a maximum
population density of only 3,662 people per square kilometre (Melbourne
10,827; Sydney 14,747) (ABS, 2015, 2011; Kennewell and Shaw, 2008). The
pressures from this extensive low density and outward expansion has
induced non-strategic and car centric development that has the potential
to degrade social and environmental systems such as amenity servicing
efficiency and habitat loss (Dhakal, 2014; Downs, 2005; Turner et al.,
2010). The guide for the long term (2050) development of Perth specifies
that future land rezoning must be the result of strategic urban planning
as opposed to individual requests (Western Australian Planning
Commission, 2015a). However, current urban monitoring within the PMR is
based upon unrepresentative (e.g.~land value information) and temporally
varied data. Thus, owing to PMR's vast and rapid expansion alongside a
globally diverse range of urban characteristics (e.g.~compact central
business district, older residential areas and new suburban
developments) it provides a timely and relevant example to develop
innovative solutions in determining temporally consistent urban area
models whilst remediating frequently reported over (or under) urban
estimation from medium resolution imagery.

In a similar theme under the sustainability key strategy within the long
term development guide, recently devised metropolitan and local
temperature mitigation plans use limited and/or aggregated data which
could misguide remedial actions (Western Australian Planning Commission,
2012). Based on the lack of global investigation into the causes and
consequences of the metropolitan and local UHI effect alongside Perth's
large scale conversion of natural to impervious surfaces it provides a
unique and globally important case study in resolving current
limitations of expanding urban areas and their association to
temperature. More specific information pertinent to each objective can
be found within each paper style chapter, with further contextual study
area information provided in chapter 4.

\section{Thesis structure and methodological
outline}\label{thesis-structure-and-methodological-outline}

This thesis is composed of nine chapters including two introduction
chapters, this aim and objectives chapter, four chapters presented in
the form of scientific journal articles aligning with the `three-paper'
format PhD submission, a discussion and final conclusion chapter. Each
paper style chapter is taken from a published or submitted journal
article, with minor editing to ensure consistency throughout the thesis.
Whilst each paper explores unique research aims, Figure 3 1 outlines
inter-linkages between the collective papers forming a coherent body of
novel research contributions. Specifically temporal land cover data
produced within the first paper is used within paper 2 and 3, with
analysis from paper 3 informing paper 4 (Figure 3 2).

Figure 3 1. Outline of general thesis structure and paper linkages.

Figure 3 2. Data flow and linkage throughout papers presented in thesis.

Following this thesis introduction and literature review chapter 4
provides an applied example into using EO data for monitoring urban
expansion in the city of Perth, WA. The implications of a rapidly
expanding urban area are discussed and EO derived estimates are compared
to those provided by the WAPC based upon temporally varied
methodologies. This study provides the first EO temporal examination of
land cover within Perth from normalised satellite imagery highlighting
the applicability of EO data in accurate urban quantification for
sustainable targeted planning practices and addressing the first thesis
objective.

Building on the work in chapter 4, chapter 5 further analyses classified
hard and sub-pixel 2007 land cover data in relation to classified high
resolution (20 cm) aerial imagery. The hard classification refers to the
dominant land cover within each Landsat pixel, whilst the sub-pixel
classification represents the probability of a pixel containing a
classification value. High resolution imagery was classified using
Object Based Image Analysis (OBIA) and aggregated to Landsat pixels (30
m) producing the percentage of land cover and identification of the
dominant land cover class per Landsat pixel area. Firstly, the two hard
land cover datasets were compared, identifying overestimation from the
Landsat classification. Addressing errors of this nature are essential
owing to EO land cover data being used to influence policy decisions.
Overestimation was remediated through the implementation of novel
spatially explicit regression model approach between the high resolution
percentage urban per Landsat pixel and the Landsat sub-pixel data that
improved urban land cover estimations from medium resolution imagery,
achieving the second thesis objective.

Chapter 6 uses 2003 and 2013 classified land cover data produced in
chapter 4 alongside temperature data from the MODIS Terra sensor. The
methodology overcomes current limitations of UHI studies such as use of
temporally static land cover, assumption of urban homogeneity and
disregard of a spatial component through global indices, inappropriate
for policy incorporation. Land cover estimates were aggregated to MODIS
resolution (1 km) producing the percentage of land cover per MODIS pixel
for both 2003 and 2013. The dominant land cover change per MODIS pixel
was identified and associated with the difference in temperature between
2003 and 2013. Consequently the presented novel analysis established
ideal future land rezoning in relation to temperature change,
accomplishing the third thesis objective.

Chapter 7 draws upon chapter 6 analysis in demonstrating an improved
localised UHI mitigation approach. Current global, metropolitan and
localised mitigation policies and strategies often fail in planning
practicality through a lack of specificities or inappropriate data use.
Chapter 7 demonstrates the power of EO data and spatial modelling in
reducing localised temperature for a proposed redevelopment in the City
of Fremantle, WA through optimum vegetation placement using a
scientifically valid and policy integratable approach, advancing current
policy and academic mitigation attempts, aligning with the final
objective of this thesis.

Chapter 8 critically discusses the research significance, methodological
transferability, global applicability, current critical challenges and
future research potential.

Chapter 9 provides a summary of the key findings of the thesis in
relation to the overall thesis aim.

\chapter{Urban growth dynamics in Perth, Western Australia: using
applied remote sensing for sustainable future
planning}\label{urban-growth-dynamics-in-perth-western-australia-using-applied-remote-sensing-for-sustainable-future-planning}

\section{Abstract}\label{abstract-1}

Earth observation data can provide valuable assessments for monitoring
the spatial extent of (un)sustainable urban growth of the world's cities
to better inform planning policy in reducing associated economic, social
and environmental costs. Western Australia has witnessed rapid economic
expansion since the turn of the century founded upon extensive natural
resource extraction. Thus, Perth, the state capital of Western
Australia, has encountered significant population and urban growth in
response to the booming state economy. However, the recent economic
slowdown resulted in the largest decrease in natural resource values
that Western Australia has ever experienced. Here, we present
multi-temporal urban expansion statistics from 1990 to 2015 for Perth,
derived from Landsat imagery. Current urban estimates used for future
development plans and progress monitoring of infill and density targets
are based upon aggregated census data and metrics unrepresentative of
actual land cover change, underestimating overall urban area. Earth
observation provides a temporally consistent methodology, identifying
areal urban area at higher spatial and temporal resolution than current
estimates. Our results indicate that the spatial extent of the Perth
Metropolitan Region has increased 45\% between 1990 and 2015, over 320
km2. We highlight the applicability of Earth observation data in
accurately quantifying urban area for sustainable targeted planning
practices.

\section{Introduction}\label{introduction-3}

Over the last 15 years, Perth has experienced exponential economic
growth with Gross State Product (GSP) increasing 218\% (ABS, 2015).
Originally labelled as the `Cinderella State' due to its remote location
and perceived neglect from the rest of Australia, Western Australia (WA)
has experienced sustained discovery and extraction of natural resources
since the beginning of the 21st century (Kennewell and Shaw, 2008). In
response to a growing resource sector, the city of Perth has undergone
extensive urban expansion at what Dhakal (2014) identified as an
unsustainable rate. To this end, the Western Australian Planning
Commission (WAPC) identified that Perth's urban footprint has increased
from 631 km2 to 870 km2 in the 10 years between 2002 and 2012 (Western
Australian Planning Commission, 2010a, 2015a). However, these figures
should be considered with caution as data used in early estimates
represent land parcel (Cadastral) valuations only (provided by the
Western Australian Value General's Office), with later estimates (from
2009) based on multiple urban zoning classifications, and more recently
(from 2010) spatial modelling taking into account land valuation and
zoning (Western Australian Planning Commission, 2010b, 2009). The use of
varied data and methods impacts confidence in the ability of the
Commission's estimates to represent actual change in urban extent,
especially when urban zoning information includes land identified for
growth but not necessarily developed. Such inconsistencies could have
potential to misinform future development decisions. Consequently, here
we present a spatiotemporal assessment of change in areal urban growth
based upon medium resolution remote sensing through a single
classification model. This provides the first accurate depiction of
urban expansion for one of the world's fastest growing cities---Perth,
WA. We present our findings and discuss the implications of more
accurately classified urban extents in facilitating scientifically
evidence-based adaptive and targeted planning policies to help reduce
environmental and socio-economic consequences of poorly planned
development.

\subsection{Earth observation for monitoring urban
change}\label{earth-observation-for-monitoring-urban-change}

Mapping the spatial extent and temporal profile of urban growth from
medium resolution satellite imagery facilitates a consistent, detailed
characterisation of the actual urban footprint of a city (Angiuli and
Trianni, 2013; Bagan and Yamagata, 2012b). Other conventional spatial
datasets such as Cadastral data provide information on freehold and
Crown land parcel boundaries including attributes such as ownership and
value for a singular temporal period (Thompson, 2015). However,
attributed data for a singular year provides an ineffective portrayal of
actual parcel land cover and temporal change. Thus, the methods and
results presented in this study provide foundational information for the
development of planning regulations that ensure sustainable growth of
our cities, particularly in the reduction of environmental risks from
ever-increasing expansion along the wildland--urban interface (Turner et
al., 2010). Specifically, Earth Observation (EO) data allows spatially
detailed identification of locations where (un)sustainable urban growth
is occurring which enables expansion limits to be imposed through
targeted policies (Bettencourt and West, 2010). In this theme, Schneider
et al. (2005) determined the spatial distribution of development zones
from 1978 to 2002 in Chengdu, Sichuan province, China in response to the
Go West policy of the 1990s, aimed at economically boosting the West of
the country. Whilst the policy was successful in raising Gross Domestic
Product (GDP) levels, urbanisation concurrently increased, generating
issues of urban management, including service, infrastructure and
resource deficiency. Their results indicated spatial clustering,
specialisation of land use and peri urban development (not considered by
the original policy) which were subsequently used to tailor policy in
remediating issues, facilitating sustainable future urban development
(Patino and Duque, 2013; Schneider et al., 2005). Similarly,
Hepinstall-Cymerman et al. (2013) used classified Landsat data to
monitor urban growth in regards to imposed growth boundaries in the
Central Puget Sound, Washington, United States of America. Surprisingly,
more new development occurred outside the growth boundaries than inside
within their last time period, illustrating the ineffectiveness of the
imposed policy leading to economic and ecological consequences,
including a loss of avian diversity in native forest species
(Hepinstall-Cymerman et al., 2013; Hepinstall et al., 2008). These
studies highlight the potential effectiveness of EO data in consistently
monitoring the spatiotemporal dynamics of urban development for applied
policy outcomes and ensuring sustainable future planning decisions, for
which such outputs are unachievable from traditional datasets.

\subsection{The case of Perth}\label{the-case-of-perth}

Perth's dramatic urban expansion can be attributed to Australia's
minerals and energy boom commencing at the turn of the century.
Queensland (QLD) and WA were at the forefront of the boom contributing
the largest proportion of the nation's resources output, valued at 3.3\%
of GDP (ABS, 2015). In WA, mining and petroleum extraction dominate
exports, peaking at 95\% of the state's export earnings between 2010 and
2011 (Department of Mines and Petroleum, 2015). The increase in
extraction was predominantly attributable to greater demand for raw
materials from China, resulting in steady growth of the WA mineral and
petroleum industry from AUD 4.7 billion in 1996 to a peak of AUD 121.6
billion mid-2013. However, in 2009, a 10.3\% reduction in the overall
value of mineral and petroleum resources resulted from falling commodity
prices and the 2007--2009 global financial crisis (Department of Mines
and Petroleum, 2015). Again in 2012, a further 9\% reduction in resource
value was observed as uncertainty in global economic conditions
increased (Department of Mines and Petroleum, 2015). The largest decline
to date occurred between 2014 and 2015, with an additional 22\%
reduction in the value of mineral and petroleum resources as a result of
surplus capacity, decreased demand, and decline in the value of the
Australian dollar (Department of Mines and Petroleum, 2015). The
temporal trend in resource value indicates a stagnation and decline
since late 2013 (Figure 4 1).

Figure 4 1. Timeline of natural resource value (based on Department of
Mines and Petroleum annual reports) fitted with a fourth order
polynomial trend line and population (based on Australian Bureau of
Statistics data) also indicating key milestones.

Perth is described as one of the most isolated cities in the world (pop.
\textgreater{} 1 million) and was Australia's fastest growing metropolis
between 2007 and 2014; however, subsequent to a decline in natural
resource value, a slowdown in population expansion soon followed (Figure
4 1) (Kennewell and Shaw, 2008). As a result, 2015 population statistics
highlight the lowest population increase since records began with a
0.5\% increase from the previous year (ABS, 2015; Kennewell and Shaw,
2008). In comparison to other Australian state capitals, based on the
Australian Bureau of Statistics (ABS) 2011 population grid, Perth
exhibits a relatively sparse spatial distribution of population with a
maximum population density of only 3662 people per square kilometre
(Melbourne 10,827; Sydney 14,747). Such low density population has
generated high demand for dispersed housing, amenities and services, and
has influenced changes to Perth's land use patterns in a non-strategic,
``lot-by-lot fashion'' based on a car-dependent lifestyle (Dhakal,
2014). Anthropogenic modifications of the landscape from vegetation
cover to human-made impervious surfaces represent a critical driving
force in both local and global environmental change (Kalnay and Cai,
2003; Vitousek et al., 1997). For example, abrupt, poorly planned and
uncontrolled urban expansion can lead to environmental impacts which
degrade ecological systems including habitat fragmentation and
socio-economic issues that deteriorate efficiency of amenity
provisioning, both of which can exacerbate localised climate change
(Downs, 2005; Turner et al., 2010). Identifying impacts of Land Use and
Land Cover (LULC) change on socio-ecological systems is vital for future
sustainable urban development; as reflected in the ``sustainable cities
and communities'' 2030 sustainable development goal and the effective
land use planning criteria of the City Resilience Framework (CRF) (ARUP
and The Rockefeller Foundation, 2015; Vitousek et al., 1997). It is
essential for Perth to adapt current practices of outward suburban
expansion to achieve more sustainable urban growth and become city-smart
for accommodating the predicted additional half a million new residents
by 2031, which will result in an overall population exceeding 2.2
million (Western Australian Planning Commission, 2010a).

\section{Materials and methods}\label{materials-and-methods}

\subsection{Data preprocessing}\label{data-preprocessing}

EO data have been extensively used to monitor the sustainability of
urban areas (Li et al., 2015; Song et al., 2016). However, accurate
identification and temporal monitoring of urban land is frequently
precluded due to the coarse resolution (300 m--1 km) of a number of
commonly used remotely sensed datasets including nighttime lights (1 km)
and the Moderate Resolution Imaging Spectroradiometer (MODIS) land cover
product (0.083°) (Potere et al., 2009; Song et al., 2016). Whilst 30 m
resolution data (e.g.~Landsat) are more suitable to detect nuances of
urban development the majority of studies and classified products which
have used these finer resolution products implement large temporal
windows, negating the possibility of detailed temporal urban
characterisation (e.g.~GlobeLand30, Hu et al., 2015; Masek et al., 2000;
Suarez-Rubio et al., 2012; Van de Voorde et al., 2011; Xian et al.,
2012). This research provides the first comprehensive temporal evolution
analysis quantifying land cover change and associated urban expansion
for the Perth Metropolitan Region (PMR) using 30 m Landsat imagery, the
longest temporal record of medium spatial resolution imagery, for seven
sequential time snapshots between 1990 and 2015.

Cloud free imagery was acquired in or close to the month of July for
1990, 2000, 2003, 2005, 2007, 2013 and 2015. Analysis of imagery
acquired from WA winter season coincided with peak green-up which
provided the greatest contrast between spectrally similar surfaces
(e.g.~bare earth and urban) (Herold et al., 2002; Lu et al., 2011;
Varshney and Rajesh, 2014). Imagery date selection was founded upon the
strong positive relationship between Australian soil moisture (related
to rainfall) and the Normalised Difference Vegetation Index (NDVI) (Chen
et al., 2014), which exhibits an approximate one month lag between peak
soil moisture and peak NDVI (Chen et al., 2014).

Productive photosynthesising plants use energy in the visible red (VIS)
portion of the electromagnetic spectrum whilst reflecting in the
near-infrared (NIR) region. NDVI ((NIR -VIS)/(NIR + VIS)) is a
representative measure of growth allowing for the identification of
green, healthy vegetation (Chen et al., 2014; Myneni et al., 1997; Piao
et al., 2011), as illustrative of Southwest WA's winter months. A total
of 14 images from Landsat 5 Thematic Mapper (TM) (eight images), Landsat
7 Enhanced Thematic Mapper Plus (ETM+) (two images) and Landsat 8
Operational Land Imager (OLI) (four images) were acquired for the
specified years. Seamless images were produced based on Voroni diagrams
that locate the bisector between images; adjacent edges were identified
as seamlines constraining effective mosaic polygons that specify
inclusion pixels for the final mosaicked product, permitting less
visible boundaries through blending overlapping pixels (Pan et al.,
2009). Mosaicked images were subsequently clipped to the original PMR
study area boundary.

The atmospherically corrected Landsat data used in this study were
obtained from the Landsat Ecosystem Disturbance Adaptive Processing
System (LEDAPS) and the Landsat 8 Surface Reflectance (L8SR) algorithm
(Hansen and Loveland, 2012; USGS, 2015). Some inherent residual noise
remained, for example, due to the differences in modelled atmospheric
correction parameters (Ju et al., 2012). To correct for this, surface
reflectance values were standardised as:

p\_(i,b)=p\_(x,b)/〖max〗\_b (4.1)

where p\_(i,b) is the standardised pixel value i, from band b based on
the original surface reflectance x, standardised through division of a
priori specific upper reflectance limit for each band (maxab): 0.1
(blue; 0.48 µm), 0.11 (green; 0.56 µm), 0.12 (red; 0.66 µm), 0.225
(near-infrared; 0.84 µm), 0.205 (shortwave-infrared; 1.65 µm), 0.150
(shortwave-infrared 2; 2.22 µm) (Sexton et al., 2013). Standardised
values were then normalised per pixel j through cross band sum division:

p\_(j,b)=p\_(i,b)/(∑\emph{i▒p}(i,b) ) (4.2)

where ∑\emph{i▒p}(i,b) is the sum of each standardised pixel across all
bands (Sexton et al., 2013). Normalised Landsat data obtained a
statistically significant reduction of spectral variation per land cover
class within (inter) and between (intra) each image (see Supplementary
Figure 4 5).

\subsection{Data classification}\label{data-classification}

The normalised Landsat imagery was classified using the Import Vector
Machine (IVM) which builds upon the popular Support Vector Machine (SVM)
methodology (Roscher et al., 2012). In order to obtain the optimum
classification, the IVM algorithm explores all possible subsets of
training data for optimal selection (termed import vectors) which are
derived through successively adding training data samples until a given
convergence criterion is met (Roscher et al., 2012). Data samples are
selected according to their contribution to the classification solution.
However, a pure forward system is unable to remove import vectors that
become obsolete after addition of other vectors. Therefore the
implemented version of IVM utilised here is a hybrid forward/backward
strategy that adds import vectors whilst concurrently testing if they
can be removed in each step, thus leading to a sparse and more accurate
solution (Roscher et al., 2012). Furthermore, the IVM selects data
points from the entire distribution resulting in a smoother decision
boundary which is based on the optimal separating hyperplane in
multidimensional space compared to that of SVM algorithms (Braun et al.,
2012). The benefits of the IVM algorithm have resulted in this approach
being successfully applied in a number of studies (e.g.~Braun et al.,
2012, 2011; Roscher et al., 2010; Suess et al., 2014) due to its
accuracy and performance advantages over alternative methodologies
including SVM and the traditional Maximum Likelihood (ML) classifier
(Braun et al., 2011; Roscher et al., 2010).

Model training samples were selected using the July 2005 Landsat 5 TM
image coinciding with the month post maximum rainfall of all considered
Landsat 5 TM and 7 ETM+ to facilitate optimum spectral seperability
(Chen et al., 2014). Land cover was defined as high albedo urban
(e.g.~concrete), low albedo urban (e.g.~asphalt) or other. Two urban
classes were initially identified in order to reduce confusion between
spectrally similarly classes (e.g.~urban and bare earth) being merged
post-classification to represent complete urban coverage (Hu and Weng,
2009). For each class, 250 pixels were randomly selected as training
data, which is consistent with Foody and Mathur (2006) and Pal and
Mather (2003) (see supplementary section 4.7.2). Training data
parametrised the IVM algorithm, creating a classification model of
spectral profiles that are compared to Landsat spectral profiles for
classification. The classification model was then applied to all Landsat
5 TM and Landsat 7 ETM+ images obtaining similar spectral wavebands,
considered to be equivalent (Flood, 2014). However, due to Landsat 8 OLI
sampling different spectral regions, a new classification model was
developed using the same training areas, as these were deemed to remain
representative of the land cover, but with Landsat 8 OLI spectral
wavebands (Flood, 2014; Roy et al., 2016). Validation was performed
through an accuracy assessment based on an independent dataset (Google
Earth high resolution imagery) consistent with Landsat acquisition
months following previously published methods (e.g.~Bagan and Yamagata,
2014; Cunningham et al., 2015; Dorais and Cardille, 2011; Song et al.,
2016; Sun et al., 2015; Zhu and Woodcock, 2014). For each land use
category, 50 pixels per class per year were visually identified and
classified based on the majority land cover within the coincident
Landsat pixel from Google Earth imagery for the available years 2000,
2003, 2005, 2007, 2013 and 2015 consistent with recommended land cover
accuracy sample size of Congalton (2001).

\section{Results}\label{results}

The spatial footprint of PMR development has increased 48.61\% between
1990 and 2015, over 320 km2 (Figure 4 2 and Figure 4 3), with a 40.56\%
increase occurring since 2000. The classification accuracy assessment
indicates an average overall accuracy of 81.06\% and Kappa Coefficient
of 0.73 being comparable to other studies (e.g.~Bagan and Yamagata,
2014; Gislason et al., 2006; Luo et al., 2014; Sundarakumar et al.,
2012) (see Supplementary Table 4 1 and Supplementary Table 4 2). Urban
expansion mirrors population increase and as population growth has
slowed, urban development has concurrently exhibited a levelling trend
compared to expansion previously observed (Figure 4 3).

Figure 4 2. Urban expansion within the Perth Metropolitan Region (PMR)
between 1990 and 2015. Vast urban growth has been observed in PMR with
graduating colours exhibiting outward expansion (a); (b) and (c) exhibit
static snapshots of urban extent from 2000 (b) and 2015 (c); whilst (d)
depicts percentage of urban change per subnational administrative
boundary (Local Government Area (LGA)).

Figure 4 3. Time line of urban expansion in kilometers squared derived
from Earth observation data with associated classification error derived
from validation data (points indicating classified image years).
Alongside population data in millions per year since 1988 (based on
Australian Bureau of Statistics data, 2015 data is projected) with key
natural resource milestones indicated, and average annual urban and
average annual population growth rate indicated between classified image
years.

WAPC's urban estimates of the PMR from Directions 2031 (the strategic
plan for the Perth and Peel region) were provided for comparison to
those produced within this study (Western Australian Planning
Commission, 2010a). WAPC's estimates note an expansion from 637 km2 to
813 km2 between 2001 and 2012. Our results indicate an expansion of
747.41 km2 to 1050.57 km2 from 2000 to 2013 illustrating an overall
underestimation by WAPC figures (Figure 4 4). Within suburban areas
surrounding the two major cities in the metropolitan region, Perth and
Fremantle, WAPC's estimates underrepresent the amount of urban area
derived from EO, being more pronounced in 2013 than 2000. The Local
Government Area (LGA) of Stirling South Eastern (LGA outlines displayed
in Supplementary Figure 4 6) represented the maximum overestimation in
2013 urban area with 34.47\% (2000: 9.95\%) additional urban area per
km2 of LGA established on a difference of 2.89 km2, 41.91\% (2000: 0.83
km2, 14.99\%) between EO data and WAPC's estimates. Outer Northern and
Southern LGA WAPC urban values were consistently underestimated, with
the LGA of Belmont representing the maximum underestimation of percent
per km2 of LGA in 2013 with 23.62\% (2000: 12.60\%) due to a difference
of 9.37 km2, 40.39\% (2000: 5.00 km2, 26.46\%). Prior to 2009, WAPC's
estimates were solely based upon land parcel valuations from the Western
Australian Value General's Office, consequently valuation thresholds
designating land to urban may have been inappropriately applied to outer
suburban LGAs, where land might be developed but less valuable than
central LGAs.

For urban estimates post 2005, two urban land zones, urban and urban
deferred, are used within the Perth Metropolitan Region Scheme (MRS),
the division of the State Planning Policy Framework applicable to the
PMR, pursuant to the Planning and Development Act (2005) that inform
recent WAPC land parcel based estimates (Western Australian Planning
Commission, 2016b, 2010c). Urban land refers to locations where
activities in line with urban development are permitted, but not
necessarily constructed (e.g.~housing and commercial use) whilst urban
deferred represents land suitable for future development with remaining
planning, servicing or environmental issues (Western Australian Planning
Commission, 2016a, 2010c). For land to be assigned urban deferred, it
must obtain characteristics of the urban zone including being able to
provide essential services, a logical progression of development, and
able to satisfy regional requirements (e.g.~roads and open spaces). The
2012 WAPC estimates were derived from stock of land zoned urban or urban
deferred, cadastral land plot and value information, conditional
subdivision approvals, and ongoing regional rezoning and subdivisions
(Western Australian Planning Commission, 2012). Similarly to 2000,
valuation data may misrepresent suburban urban land cover resulting in
overestimation. Inclusion of additional variables that are
unrepresentative of actual land cover change (e.g.~rezoning and
conditional approvals) could exacerbate differences between WAPC and EO
derived urban estimates (Figure 4 4 (b)), leading to the potential
confounding of errors in WAPC estimates.

Figure 4 4. Percentage differences relative to local government area
size, permitting a change metric standardised by Local Government Area
(LGA) area between Earth Observation (EO) and the Western Australian
Planning Commission's (WAPC) urban estimates for (a) 2000 (EO) and 2001
(WAPC) and (b) 2012 (WAPC) and 2013 (EO), whilst (c) depicts the
percentage difference in the relative urban rate of change (km2 per LGA
area) between 2000 and 2013 (EO) and 2001 and 2012 (WAPC). Positive
values indicate underestimation by WAPC whilst negative values represent
overestimation by WAPC.

\section{Discussion}\label{discussion}

WA state government planning documentation states that the majority of
new development within the PMR has occurred as low-density suburban
growth, responding to consumer preferences and market forces (Western
Australian Planning Commission, 2015a). Additionally, sustainable policy
objectives suggest that new development should be managed and focused on
current communities, making the most efficient use of existing urban
areas (Western Australian Planning Commission, 2010a, 2015a). Planning
policy research has highlighted issues of outward urban expansion as
being costly in economic, environmental and social terms based on
dispersed service requirements, habitat fragmentation and neighbourhood
segregation (Downs, 2005). Thus, urban expansion in the PMR may result
in further economic, social and environmental costs associated with
servicing and maintaining low-density lifestyles, owing to the rapid
outward urban growth estimates between 2000 and 2007 (Downs, 2005;
Turner et al., 2010).

In contrast, the witnessed slowdown of urban growth, population and
natural resource value since 2013 indicates the possibility that the
`boom' of previous years has reached a turning point. Stagnation of
urban growth implies that issues associated with spatially distributed
urban areas might be contained to the current urban extent.
Nevertheless, it is conceivable that prosperous future economic
circumstances could initiate growth at a rate previously observed, and
that the economic slowdown might be a temporary hiatus responding to
current economic conditions (Perry and Rowe, 2014). For example, in
2014--2015, WA continued to attract the largest proportion of state
mineral exploration expenditure at 58\%, with QLD (the second ranked
stated) obtaining only 20\% (Department of Mines and Petroleum, 2015).
Furthermore, as of September 2015, WA had an estimated AUD 171 billion
in mineral and petroleum projects under construction, with a further AUD
110 billion allocated for future expansion (Department of Mines and
Petroleum, 2015). Comparatively, during the peak (mid-2013) in terms of
total sales, WA only had an estimated AUD 160 billion worth of projects
under construction and a further AUD 108 billion for future development
(Department of Mines and Petroleum, 2015). Whilst 2014--2015 observed
the greatest decline in total sales of resources, sustained investment
and improved global economics could reinvigorate the industry and
reinitiate urban expansion within the PMR.

Future development (urban and urban deferred) is guided by Directions
2031 amending the MRS and local planning schemes (Western Australian
Planning Commission, 2015b, 2015c, 2015d, 2015e, 2010a). WAPC aims to
achieve 47\% of future development as infill and a 50\% increase in
average residential density by 2050 of 10 dwellings per urban zoned
hectare and 15 per new urban zoned hectare (Western Australian Planning
Commission, 2010a). In monitoring progress towards the infill target,
zoned development land within the PMR is considered, including
residential, industrial and commercial land uses (Western Australian
Planning Commission, 2015a). Densities are defined as infill or
greenfield if above or below an undocumented residential threshold from
census data (Western Australian Planning Commission, 2016a). Initial
results from Delivering Directions 2031, 2014 report indicate the
requirement of a significant increase in infill development if the above
targets are to be met (Western Australian Planning Commission, 2014).
Similarly, average residential density monitoring has been achieved with
land valuation data (from the Valuer General's Office) for major
activity centres, being unrepresentative of actual density change and
providing an incomplete metropolitan comparison (Western Australian
Planning Commission, 2014). Inclusion of EO data would permit
quantitative evidence of urban expansion, infill and density at a higher
spatial and temporal frequency than current census based estimates. This
would facilitate credible, evidence-based efficient targeted action
founded upon improved representative urban area, insuring infill and
density attainment. In this theme, Schneider et al. (2005) and
Hepinstall-Cymerman et al. (2013) used spatial metrics (e.g.~urban area
mean patch size) based on classified Landsat data in either pre-defined
census units (Hepinstall-Cymerman et al., 2013) or development corridors
(Schneider et al., 2005) to monitor development type (infill or
expansion) over time for adapting inappropriate static urban development
policy. Using EO derived land cover data in this manner aids in
understanding dynamics of the urban environment through monitoring,
planning and mitigating land use changes that impact natural assets and
increase vulnerability of city systems (Hepinstall-Cymerman et al.,
2013; Miller and Small, 2003; Patino and Duque, 2013). Information of
this sort aligns with criteria of the CRF in improving city resilience
from effective land use planning, possible at lower expense and higher
temporal frequencies than in situ measurements (ARUP and The Rockefeller
Foundation, 2015).

The universal methodology implemented within this research lends itself
to credibly inform policy in a similar manner in other global cities
through monitoring urban expansion in order to identify rapid,
unsustainable development. For example, Jakarta, obtaining the world's
second largest urban area with a population of 28 million, has yet to
have any quantitative urban area delineation (Pravitasari et al., 2015;
Seto et al., 2011). Identification of actual urban growth in developing
cities is vital to city planners, environment managers and policy makers
due to the difference between planned growth and actual growth (Patino
and Duque, 2013). Such information could be of critical importance for
regulating urban expansion due to extreme poverty and high level of risk
to environmental hazards, such as that posed from flooding (Marfai et
al., 2014; Suryahadi and Sumarto, 2003). EO data presents many
opportunities for added value within urban planning policy, and
additional analyses could be pursued into specific human-induced
environmental issues, such as detecting thermal changes in the urban
environment for planning issues associated with urban heat islands
(e.g.~cooling provisions) and their impact on human health (e.g.~air
quality).

\section{Recommendations}\label{recommendations}

Consistent and accurate LULC estimates are a vital aspect of sustainable
urban development throughout the world, especially considering the
predicted additional 2.5 billion city dwellers by 2050. LULC models that
require agents that are representative of land use decisions can often
fail in practicality due to the difficulty in quantifying driving forces
of change and multi-level relationships. Models of this nature are also
temporally independent, with each annual iteration implementing new data
or data not representative of actual LULC change. EO data provides a
replicable detailed representation of the complete urban extent
requiring no additional data. The use and application of EO data
reported within this paper highlights several improvements to WAPC
policy for consistent urban area estimations with associated accuracy
measures. Therefore it is recommended that planning authorities, such as
WAPC, integrate EO data to achieve the following: (1) provide scientific
urban estimates based on a temporally consistent model within future
regional structure plans, metropolitan region and local planning
schemes; (2) monitor infill development at a higher temporal frequency
than census years for policy targeting to meet key goals; (3) monitor
urban density through areal urban expansion compared to current metrics
using land valuations; and (4) restrict development based on temporal
urban analysis that degrades amenity efficiency and ecological systems
whilst promoting development in locations to maximise efficiency and
long-term sustainability. Additional EO datasets (e.g.~finer resolution
Sentinel 2 satellite imagery or aerial imagery) could be used to refine
planning decisions based on areas of concern identified from Landsat.
For example, finer spatial resolution datasets could facilitate enhanced
feature extraction, optimising sustainable planning decisions through
the identification of candidate infill sites. EO data of this nature
provides an essential tool for timely planning policy that is adaptive
to changes in urban landscape to mitigate socio-environmental issues
associated with poorly planned urban areas for the future sustainability
of our cities.

\section{Supplementary material}\label{supplementary-material}

This supplementary material section supports the main chapter and thesis
through providing further detail into the summarised methods and
results, pertaining to three concepts, namely the: standardisation and
normalisation methodology, accuracy assessment and locations of Local
Government Areas (LGAs). In the first instance a statistical comparison
between the original un-corrected and corrected imagery is provided to
highlight the reduction in inter and intra-year variance for land cover
types demonstrating the validity of a single classification model for
multiple years of imagery. Whilst key overall accuracy metrics are
provided within the main chapter this supplement provides additional
measures for each classified image for appropriate contextualisation if
used by other researchers (accessible online). Similarly, Supplementary
Figure 4 6 supports Figure 4 4 and text comparing estimates produced
within the main chapter to that of the Western Australian Planning
Commission (WAPC) through visually identifying the LGAs. A short
discussion surrounding other considered approaches and a list of data
used within the chapter are also provided to demonstrate the entirety of
analytical process.

\subsection{Standardisation and
normalisation}\label{standardisation-and-normalisation}

The non-urban land cover class was composed of four classes defined
based on existing literature (e.g.~Feyisa et al., 2016; Hu and Weng,
2009; Schneider, 2012) and study area characteristics. These classes
were forest, water, grassland and bare earth. Grassland and forest were
also merged to create a single vegetation class. A complete set of land
cover classes enabled examination of statistical differences generated
from normalisation (Supplementary Figure 4 5). The Coefficient of
Variation (CV) statistic was calculated to describe the amount of
variability relative to the mean spectral reflectance of the post
classification datasets (Brown, 1998). The CV was calculated for pre-
and post-normalisation datasets for both intra-year class reflectance,
which describes the variability within each class per year, and the
inter-year reflectance, which describes the variability of each class
across all imagery dates. Post-normalised Landsat data exhibited
statistically significant lower inter- and intra-CV with T = 0, Z =
−2.154, p = 0.016, r = −0.359 and T = 0, Z = −2.418, p = 0.008 r =
−0.373, respectively. The test statistic (T) was obtained from dataset
differencing (pre- minus post-processed images), representing the lowest
value of the sum of positive ranks (values increased) or negative ranks
(values decreased). Hence, T = 0 dictates that post-processed data
consistently obtained a lower value than pre-processed imagery,
statistically significant at p \textless{} 0.05 (Field, 2009).
Therefore, reduced intra and inter year variance facilitates more
appropriate one model classification for Landsat 5 TM and Landsat 7 ETM+
and another for Landsat 8 OLI.

Supplementary Figure 4 5. Inter year classification reflectance
variation categorised by classified output for each spectral band for:
pre (a) and post (b) normalisation correction.

\subsection{Accuracy assessment}\label{accuracy-assessment}

For each land use category, 50 random pixels per class per year were
visually identified and classified based on the majority land cover
within the coincident Landsat pixel from Google Earth imagery for the
available years: 2000, 2003, 2005, 2007, 2013 and 2015 (Congalton,
2001). Both user's accuracy (fraction of correctly classified pixels
relative to all others classified as a particular land cover),
producer's accuracy (fraction of correctly classified pixels compared to
ground truth data) and associated Kappa coefficients were consistently
high except for the producer accuracy of bare earth which has an average
accuracy of 53.33\% (Supplementary Table 4 1 and Supplementary Table 4
2). This is due to the known spectral similarities between bare earth
and impervious surfaces, and water and shadow which resulted in spectral
confusion during classification (Feyisa et al., 2016; Herold et al.,
2002; Lu et al., 2011; Sawaya et al., 2003; Varshney and Rajesh, 2014).

Supplementary Table 4 1. Classification accuracy and associated Kappa
Coefficient per year of classified Landsat. Year Accuracy (\%) Kappa
Coefficient 2000 82.33 0.75 2003 80.33 0.72 2005 82.00 0.74 2007 84.00
0.78 2013 79.00 0.70 2015 78.67 0.70 Average 81.06 0.73

Supplementary Table 4 2. Producer's and User's accuracy per year of
classified Landsat imagery. Producer Accuracy Bare Earth Vegetation
Urban Water 2000 56.00 96.00 90.00 66.00 2003 50.00 97.00 85.00 68.00
2005 52.00 98.00 86.00 72.00 2007 48.00 98.00 83.00 94.00 2013 52.00
99.00 81.00 62.00 2015 62.00 99.00 80.00 52.00 Average 53.33 97.83 84.17
69.00 User Accuracy Bare Earth Vegetation Urban Water 2000 84.85 76.80
84.11 94.29 2003 69.44 76.98 83.33 94.44 2005 81.25 78.40 81.13 97.30
2007 68.57 80.33 87.37 97.92 2013 68.42 73.88 83.51 100.00 2015 75.61
73.88 83.33 89.66 Average 74.69 76.71 83.80 95.60 Local government areas

Supplementary Figure 4 6. Local Government Areas (LGAs) located in Perth
Metropolitan Region (a); with (b) exhibiting LGAs South and West of the
Swan River and (c) LGAs North and East of the Swan River.

\subsection{Other considered data and
approaches}\label{other-considered-data-and-approaches}

In classifying the Landsat data a multitude of alternative approaches
were attempted before arriving at one detailed within the main chapter.
These included the use of different pre-processing data methodologies,
classifiers and data combinations. Originally raw uncorrected (digital
number) Landsat data was corrected using the Atmospheric Radiometric
Calibration of Satellite Imagery (ARCSI) tool that makes use of the Py6S
radiative transfer model (Bunting, 2017). However, due to the lack of
availability of a number of required model parameters Landsat surface
reflectance data was preferenced for consistency and repeatability.
Numerous other classifiers and data combinations were tested including
Support Vector Machines (SVMs) (Mountrakis et al., 2011), Multiple
Endmember Analysis (MESMA) (Powell and Roberts, 2008), spectral indices
such as the Normalised Difference Spectral Vector (NDSV) (Angiuli and
Trianni, 2013) and classification of multi-temporal Landsat surface
reflectance composites (Castrence et al., 2014). Nevertheless these were
all found to produce unfavourable results in comparison to the presented
approach or were unable to compute sub-pixel estimates (required within
further chapters). Further information and discussion on these
approaches is provided within section 2.2.

In a similar theme alternative data sources were considered and tested
for classification validation such as GlobeLand30 (Chen et al., 2015)
and the European Space Agency's (ESA) Climate Change Initiative (CCI)
annual global land cover (300 m) time series (1992-2015) (European Space
Agency, 2017). However, these products are constrained by limited
temporal frequency (e.g.~GlobeLand30), inadequate resolution (e.g.~CCI),
their own internal errors (e.g.~based upon an accuracy assessment) and
selection of global reference data that could fail to represent the land
cover heterogeneity within the Perth Metropolitan Region. Consequently
this prior work and reasoning guided the analysis undertaken in chapter
4.

\subsection{Chapter data list}\label{chapter-data-list}

This section provides an overview of the data presented and analysed
within this chapter.

Supplementary Table 4 3. Summary of the data, sources and applications
used within chapter 4. Data Source Application Annual population data
Australian Bureau of Statistics Contextual information Annual natural
resources value The Government of Western Australia, Department of Mines
and Petroleum annual reports Contextual information Landsat surface
reflectance data, path 112 and 113, row 82 Dates and paths 1990: 14/04
(path 112), 7/05 (path 113) 2000: 23/08 (path 112), 1/10 (path 113)
2005: 12/07 (path 112), 19/07 (path 113) 2007: 6/10 (path 112), 9/07
(path 113) 2013: 22/10 (path 112), 13/10 (path 113) 2015: 9/08 (path
112), 17/09 (path 113) Sensors Landsat 5 TM: 1990, 2003, 2005, 2007
Landsat 7 ETM+: 2000 Landsat 8 OLI: 2013, 2015 United States Geological
Survey Earth Explorer system Land cover analysis High resolution data
(read only) Google Earth Accuracy assessment Comparison urban estimates
Western Australian Planning Commission Comparison to the Commission's
estimates\\
Spatial boundary outlines Australian Bureau of Statistics Comparison to
the Commission's estimates

\chapter{Subpixel land cover classification for improved urban area
estimates using
Landsat}\label{subpixel-land-cover-classification-for-improved-urban-area-estimates-using-landsat}

\section{Abstract}\label{abstract-2}

Urban areas are Earth's fastest growing land use that impact
hydrological and ecological systems and the surface energy balance. The
identification and extraction of accurate spatial information relating
to urban areas is essential for future sustainable city planning owing
to its importance within global environmental change and
human-environment interactions. However, monitoring urban expansion
using medium resolution (30-250 m) imagery remains challenging due to
the variety of surface materials that contribute to measured reflectance
resulting in spectrally mixed pixels. This research integrates high
spatial resolution orthophotos and Landsat imagery to identify
differences across a range of diverse urban subsets within the rapidly
expanding Perth Metropolitan Region, Western Australia. Results indicate
that calibrating Landsat derived sub-pixel land cover estimates with
correction values (calculated from spatially explicit comparisons of
sub-pixel Landsat values to classified high resolution data which
accounts for over (under) estimations of Landsat) reduces moderate
resolution urban area over (under) estimates by on average 55.08\% for
the Perth Metropolitan Region. This approach can be applied to other
urban areas globally through use of frequently available and/or low cost
high spatial resolution imagery (e.g.~using Google Earth). This will
improve urban growth estimations to help monitor and measure change
whilst providing metrics to facilitate sustainable urban development
targets within cities around the world.

\section{Introduction}\label{introduction-4}

Urban areas are estimated to cover only 0.5\% of Earth's surface yet are
one of the fastest growing land use per area basis (Bettencourt and
West, 2010; Schneider et al., 2010, 2009). Population growth has
resulted in increased urbanisation with 54\% of the planet's seven
billion people in 2014 residing in urban areas with an additional 2.5
billion urban dwellers projected by 2050, whilst concurrently increasing
the proportion of world's urban population to 66\% (Powell et al., 2007;
Powell and Roberts, 2010; Sexton et al., 2013; Sharifi and Lehmann,
2014; Song et al., 2016; United Nations, Department of Economic and
Social Affairs, 2014). Alteration of natural land cover to anthropogenic
impervious surfaces has been identified as the most extreme cumulative
effect of land cover change, generating numerous socio-economic
consequences including: amenity provision efficiency, ecological
degradation and the Urban Heat Island (UHI) effect (Cai et al., 2016;
Howard, 1988; Hu and Brunsell, 2015; Xie and Zhou, 2015). Accurate
information on urban land use and land cover is therefore imperative for
monitoring expansion and planning policy targeting for future
sustainable development of our cities (Bettencourt and West, 2010; Wu
and Murray, 2003). Earth Observation (EO) enables consistent, detailed
characterisation of the actual urban footprint of a city having been
mapped and monitored using remotely sensed data at a range of spatial
and temporal scales for associated implications (Akbari et al., 2003;
Friedl et al., 2002; Imhoff et al., 1997; Schneider et al., 2010; Sexton
et al., 2013). However, accurate and consistent monitoring of urban land
cover is frequently precluded by coarse spatial (e.g.~1 km Moderate
Resolution Imaging Spectroradiometer (MODIS) land cover product) and
temporal (e.g.~2000 and 2010 GlobeLand30 product) resolution of such
datasets (Lu et al., 2014; Song et al., 2016).

Urban mapping remains challenging due to the heterogeneity of surface
materials and surface structure which contributes to pixel surface
reflectance that are often difficult to disentangle (Herold et al.,
2002; Lu et al., 2011; Schneider, 2012; Varshney and Rajesh, 2014). When
delineating urban land cover from remotely sensed data, spatial
resolution is considered the most important factor which provides
increased visibility of discrete surface features (e.g.~buildings) and
greater pixel homogeneity over medium to coarse spatial resolution
satellite imagery (e.g.~Landsat and MODIS) (Myint et al., 2011).
Nevertheless, high spatial resolution data often lack temporal
acquisition consistency (e.g.~airborne orthophotos) or are expensive to
purchase (e.g.~commercial satellite imagery). Consequently, in order to
best monitor urban land use and land cover change, datasets must have an
adequate spatial and temporal resolution to discern change. In this
regard, data from the Landsat series of satellites provides the longest
time-series of consistent, medium spatial resolution imagery that has
been extensively applied to urban area mapping (Powell et al., 2007;
Schneider and Mertes, 2014; Song et al., 2016; Sundarakumar et al.,
2012; Wilson et al., 2003; Yuan et al., 2005). Accurate quantification
of anthropogenic landscape modification is of critical importance due to
associated environmental, anthropogenic and climatic impacts (Kalnay and
Cai, 2003). Urban estimates from Landsat data have been used within
global biogeochemistry and climate models (Zhu and Woodcock, 2014),
further scientific studies such as UHI investigations (Hu et al., 2015)
and targeted urban development policies (Hepinstall-Cymerman et al.,
2013; Schneider et al., 2005). Whilst comparative studies (e.g.~Li et
al., 2014) have shown marginal holistic image accuracy difference
between algorithm selection on per-pixel Landsat classification assuming
sufficient training data. Traditional per-pixel methods, such as the
maximum likelihood classifier (discussed in supplementary section
5.9.1), have been found to significantly over or underestimate urban
area from Landsat data (Lu et al., 2011; Wu and Murray, 2003).
Addressing this error is important when accurate classifications are
required for monitoring change in land use patterns whereby calculations
of urban extent can influence decision-making (e.g.~policy for
sustainable urban development) (Bagan and Yamagata, 2014;
Hepinstall-Cymerman et al., 2013; Miller and Small, 2003; Schneider et
al., 2005). Due to the heterogeneity of urban areas, sub-pixel
classification methodologies have been increasingly applied to medium
spatial resolution data to more accurately represent the mixture of land
covers within a pixel (Lu et al., 2011; Lu and Weng, 2006; Powell and
Roberts, 2008; Wang et al., 2013; Weng and Pu, 2013). This has been
achieved through variations of Spectral Mixture Analysis (SMA) where a
set number of representative endmembers, frequently following the
Vegetation, Impervious and Soil (V-I-S) framework, are used to model the
entire image based on their spectral characteristics (Powell et al.,
2007; Ridd, 1995). However, endmembers may not fully represent image
spectral variability or a pixel may be modelled by endmembers that do
not represent materials within its field of view resulting in an
inability to adequately portray the high spectral heterogeneity of the
urban landscape (Powell et al., 2007). Support Vector Machine (SVM)
spectral unmixing attempts to resolve this issue through consideration
of a large number of training pixels which provides preferential
accuracy in comparison to SMA although high dimensional data and large
training samples can hinder its performance (Wang et al., 2013).

Comparatively the novel sub and hard pixel Import Vector Machine (IVM)
classifier permits simultaneous multi-class comparison whilst
continuously testing training samples for validity providing a more
accurate solution (Roscher et al., 2012). IVM has been found to
consistently outperform decision trees, artificial neural networks and
maximum likelihood algorithms (Huang et al., 2002; Kotsiantis et al.,
2006; Watanachaturaporn et al., 2008), with preferential (Braun et al.,
2012) and comparable results to SVM (Roscher et al., 2010). However, due
to the heterogeneity of urban areas it is important to calibrate these
sub-pixel approaches against high spatial resolution data that capture
the diverse characteristics found within urban environments (Lu et al.,
2011). Perth, Western Australia (WA) is characterised by extensive urban
diversity, surpassing all other major Australian and United States
cities in terms of suburban development (Kelly et al., 2011; U.S.
Department of Commerce, 2013). It therefore provides a suitable case
study for assessing the ability of Landsat to map urban development,
which is a pre-requisite for appropriate policy incorporation. This
paper describes an approach to map the urban extent of the Perth
Metropolitan Region (PMR) using an IVM classifier applied to medium
spatial resolution imagery. The impact of sub-pixel land cover
heterogeneity is investigated by comparing the urban area estimates to
those derived from very high spatial resolution (20 cm) imagery. An
innovative, spatially explicit correction to account for over (or under)
estimation of urban area is derived which improves the urban land cover
estimates from medium resolution imagery.

\section{Study area}\label{study-area}

The PMR (Figure 5 1 (a)), WA has experienced sustained urban development
since the 21st century in response to a rapidly growing resource sector
(Kennewell and Shaw, 2008). The majority of recent urban growth within
the PMR has transpired as outward low-density development resulting in a
maximum population density of 3,662 km2 which is 33.45\% and 24.83\%
lower than Melbourne (10,827) and Sydney (14,747) respectively (ABS,
2015; Western Australian Planning Commission, 2015a). The notion of the
`Australian dream', depicted as detached living in a green suburb, is
most pronounced in Perth (Western Australian Planning Commission,
2013a). As a result 79\% of the current housing is detached, compared to
62\% in Sydney, 72\% in Melbourne and a national average of 74\% (Kelly
et al., 2011; Western Australian Planning Commission, 2013b). Globally,
Australia surpasses other developed countries in terms of detached
suburban living with England having 42\% of housing as either detached
or semi-detached (Department for Communities and Local Government,
2015). Similarly only 64.2\% of United States of America (USA) housing
stock is detached, with Perth eclipsing all of the major 25 USA
metropolitan areas in terms of detached housing (U.S. Department of
Commerce, 2013). Low population density and outward expansion witnessed
in Perth has generated high demand for dispersed amenities and services
in a non-strategic, ``lot-by-lot fashion'' (Dhakal, 2014).
Suburbanisation of this nature has been identified as unsustainable due
to impacts on ecological systems (e.g.~habitat fragmentation) and
socio-economic issues (e.g.~amenity provisioning costs), with accurate
urban area identification essential for sustainable future planning and
maximum resource efficiency, particularly in Perth owing to its globally
high suburbanisation and distributed population (Western Australian
Planning Commission, 2013a).

Figure 5 1. Landsat 8 Operational Land Imager (OLI) true colour image
mosaic of the Perth Metropolitan Region (9 August 2015 {[}path 112{]}
and 17 September 2015 {[}path 113{]}). The locations of the high spatial
resolution aerial image subsets are indicated by coloured overlays (a),
with Western Australia identified in (b) and Perth city (c). Therefore,
the PMR provides a globally diverse range of urban characteristics
(e.g.~compact urban central business district, older residential areas
and new suburban developments) facilitating broad dataset comparison
opportunities between Landsat and high spatial resolution urban area
estimates. The high spatial resolution data identifies the complexity of
these suburban and urban areas, which is obscured in medium and coarse
spatial resolution datasets. This permits the extraction of individual
features such as buildings, roads and vegetation that compose the urban
environment and which are represented as a spectrally mixed pixel in
Landsat imagery (illustrated in Figure 5 2) (Myint et al., 2011).

Figure 5 2. Comparison of true colour high spatial resolution data (a)
(acquired from 14 March 2007) and Landsat surface reflectance (b)
(acquired on 6 October 2007 {[}path 112{]}), highlighting the spatial
detail captured by high-resolution imagery (c) and the same areas as
observed by Landsat (d) for the subset East Beechboro used within this
study. Definitive feature detection from high resolution data can assist
in refining urban area estimates produced from moderate spatial
resolution satellite imagery (Lu et al., 2011; Wu and Murray, 2003).
More accurate satellite derived urban area estimates are imperative for
ensuring appropriate data use for policy and environmental variable
applications in order to mitigate the consequences of unsustainable
urban development. This aligns with the criteria of effective land use
planning within the City Resilience Framework (CRF) which is designed to
improve city resilience (ARUP and The Rockefeller Foundation, 2015).

\hypertarget{data}{\section{Data}\label{data}}

\subsection{Landsat data}\label{landsat-data}

Cloud free Landsat scenes were obtained for 2007 from Landsat 5 Thematic
Mapper (TM), coinciding with high resolution orthophotos (described in
section 5.4.2). Imagery were acquired within winter months (9 July 2007
for path 113 and 6 October 2007 for path 112) corresponding with peak
vegetation green-up which limits issues concerning the spectral
separation between senescent vegetation, bare earth and some impervious
surfaces (Chen et al., 2014; Feyisa et al., 2016). Landsat imagery was
processed to standard terrain correction (Level 1T), geometrically and
topographically corrected using Ground Control Points (GCPs) and a
Digital Elevation Model (DEM) from the Global Land Survey 2000 dataset
(Hansen and Loveland, 2012). Landsat 5 TM surface reflectance values
were derived from the Landsat Ecosystem Disturbance Adaptive Processing
System (LEDPAS) (Hansen and Loveland, 2012; Masek et al., 2006) which
corrects for atmospheric effects using the Second Simulation of a
Satellite Signal in the Solar Spectrum (6S) radiative transfer model
(Vermote et al., 1997).

\subsection{High spatial resolution airborne
imagery}\label{high-spatial-resolution-airborne-imagery}

Radiometrically calibrated multispectral red (0.58-0.77 µm), green
(0.48-0.63 µm), blue (0.41 µm -0.54 µm) and near-infrared (0.69-1.00 µm)
orthophotos were acquired over 19 cloud free days commencing on 14 March
2007 as part of the Perth and Peel Urban Monitor Programme (Caccetta et
al., 2012). Aerial imagery, obtained between 10:00 and 14:00 to reduce
shadow effects, were captured using a Microsoft UltraCAM-D at a height
of 1300m resulting in a spatial resolution of 20 cm. Forward and side
frame overlap of 60\% and 30\% respectively permitted automatic Digital
Surface Model (DSM) extraction using geometric control points provided
by WA's land information authority (Landgate). Extraction of ground
points exclusively representing terrain variations facilitated
derivation of a Ground Elevation Model (GEM) which, when subtracted from
the DSM, generated a Relative Elevation Model (REM), depicting elevation
relative to ground points.\\
Spatial and temporal inconsistencies in reflectance can arise from
atmospheric scattering and absorption; instrument noise and
Bidirectional Reflection Distribution Function (BRDF) effects. The
latter describes the systematic variation in reflectance across an image
due to differences in view and illumination angles and which is
dependent on the surface 3D structure (Collings et al., 2011). The
orthophotos were provided as a surface reflectance product, corrected
for multiplicative and additive errors over frames (e.g.~instrument
noise and atmospheric effects) and within frame viewing and illumination
geometry (Caccetta et al., 2012; Collings et al., 2011). Image
preprocessing consisted of two steps. Firstly, a combined BRDF and
atmospheric correction procedure was applied to retrieve surface
reflectance for each image acquisition. Linear BRDF model parameters
from the Li Sparse reciprocal kernel (Wanner et al., 1995) were used to
correct for BRDF effects. Atmospheric perturbations were corrected by
assuming that the obtained digital number represented the relative
reflectance affected by spatially dependent multiplicative and additive
terms. These combined steps generated an internally consistent mosaicked
dataset. `True' surface reflectance was estimated through fitting global
offset and gain values to replicate laboratory measured calibration
targets based on the assumption that relative reflectance requires a
linear transformation to true reflectance (Collings et al., 2011).

\section{Methodology}\label{methodology}

\subsection{Landsat preprocessing}\label{landsat-preprocessing}

The two Landsat scenes covering the study area were combined to form a
seamless image mosaic following the methodology of Pan et al. (2009).
Voroni diagrams were created on the bisector between images with
adjacent edges defined as seamlines, identifying effective mosaic
polygons that specify pixels from each image to include in the final
mosaic, facilitating less visible boundaries through blending of
overlapping pixels (Pan et al., 2009) (Figure 5 1 (a)). Due to remaining
residual noise in the mosaicked imagery caused by factors such as the
brightening effect of thin clouds and atmospheric correction
differences, surface reflectance values were standardised following the
approach identified by Sexton et al. (2013):

p\_(i,b)=p\_(x,b)/〖max〗\_b (5.1)

where p\_(i,b) is the standardised pixel value i, from band b based on
the original surface reflectance x, standardised through division by a
waveband specific upper reflectance limit which are: 0.10 (blue;
0.48µm), 0.11 (green; 0.56µm), 0.12 (red; 0.66µm), 0.23 (near-infrared;
0.84µm), 0.21 (shortwave-infrared; 1.65µm), 0.15 (shortwave-infrared 2;
2.22µm). The standardised values (p\_(i,b)) were then normalised against
the summed band standardised values:

p\_(j,b)=p\_(i,b)/(∑\emph{i▒p}(i,b) ) (5.2)

where ∑\emph{i▒p}(i,b) is the sum of each standardised pixel across all
bands (Sexton et al., 2013). This approach has been found to
satisfactorily reduce variations generated from inherent residual noise
across mosaicked imagery, for example due to differences in modelled
atmospheric parameters within the LEDAPS algorithm (Luo et al., 2014;
Sexton et al., 2013) (Figure 5 3 (a)). Statistical assessment of image
radiometric normalisation provided in MacLachlan et al. (2017a) found
that the post-processed Landsat data exhibited significantly lower inter
and intra Coefficient of Variation (CV) when compared to the
pre-processed data.

Figure 5 3. Summary of classification procedures for (a) Landsat and (b)
high-resolution orthophoto data.

\subsection{Landsat classification}\label{landsat-classification}

The 2007 Landsat data was classified as a time series of data for seven
sequential periods between 1990 and 2015 using an IVM classifier
produced in MacLachlan et al. (2017a). The method uses a hybrid strategy
which assesses whether new samples (termed import vectors) can be
removed in each forward step in order to provide a smoother decision
boundary which ideally leads to a more accurate solution (Roscher et
al., 2012). Samples are selected based on how much their incorporation
decreases the objective function to minimise the decision boundary to
form the optimal separating hyperplane between overlapping clusters
(e.g.~land cover types) in spectral feature space (Mountrakis et al.,
2011; Roscher et al., 2012; Zhu and Hastie, 2005). IVM generates two
outputs, a soft (sub-pixel) dataset which defines the probability of a
pixel containing a given classification value (e.g.~land cover type) and
a traditional `hardened' classified dataset (Braun et al., 2012).
Training samples were collected from the 12th and 19th July 2005 Landsat
5 TM image composite, coinciding with peak vegetation greenness which
provides the greatest spectral separability between vegetated and
non-vegetated surfaces (Chen et al., 2014; Feyisa et al., 2016). Six
land cover types were defined based on existing literature (e.g.~Feyisa
et al., 2016; Hu and Weng, 2009; Schneider, 2012) and scene analysis
which are high reflectance urban (e.g.~concrete), low reflectance urban
(e.g.~asphalt), forest, water, grassland and bare earth. Two urban land
cover classes are specified to reduce spectral confusion between
spectrally similarly classes (e.g.~urban and bare earth) (Hu and Weng,
2009). For each land cover type, 250 pixels were randomly identified
from across the image for training the IVM classifier which follows the
approach used by Foody and Mather (2006) and Pal and Mather (2003). The
IVM algorithm is parameterised using the training data that generates a
classification model consisting of spectral profiles for each land cover
type, which are then matched to the Landsat mosaic during
classification.

The resulting per-pixel (hardened) classification indicates that the
total urban extent of the PMR has increased 45.32\% (sub-pixel estimate
of 32.96\%) between 1990 (hardened estimate 706.88 km2, sub-pixel
estimate 736.93 km2) and 2015 (hardened estimate 1027.22 km2, sub-pixel
estimate 979.84 km2) (MacLachlan et al., 2017a). This can be broken down
into low reflectance urban cover expanding from a hardened value of
592.83 km2 (sub-pixel estimate 668.46 km2) to 839.00 km2 (sub-pixel
estimate 850.87 km2) and high reflectance urban cover increasing from a
hardened value of 114.05 km2 (sub-pixel estimate 135.32 km2) to 188.20
km2 (sub-pixel estimate 214.06 km2) across the same temporal period.

\subsection{Google Earth Landsat accuracy
assessment}\label{google-earth-landsat-accuracy-assessment}

Google Earth imagery consistent with the Landsat acquisition date was
used to assess the accuracy of the hardened Landsat classification
following previously published methods (e.g.~Bagan and Yamagata, 2014;
Cunningham et al., 2015; Dorais and Cardille, 2011; Song et al., 2016;
Sun et al., 2015; Zhu and Woodcock, 2014). Using the Google Earth
imagery, 300 random locations (50 per land cover class) within the PMR
which were visually identified and compared to the classified land cover
data, consistent with recommended land cover accuracy sample size of
Congalton (2001) (Song et al., 2016). The 2007 Landsat classification
obtained an accuracy of 84.00\% and a Kappa Coefficient of 0.78. Urban
land cover estimates had a producer's accuracy of 83.00\% and user's
accuracy of 87.37\%. MacLachlan et al. (2017a) provide a full breakdown
of urban temporal change and associated accuracy for all imagery in the
Landsat time series (1990-2015), with the Landsat classification data
available from the pangaea open access publisher (DOI:
10.1594/PANGAEA.871017) (MacLachlan et al., 2017b).

\subsection{Aerial image
classification}\label{aerial-image-classification}

Urban areas are complex, heterogeneous environments which are
challenging to classify even when using high spatial resolution
multi-spectral imagery (Lu et al., 2011; Varshney and Rajesh, 2014).
Within urban areas, traditional moderate and coarse spatial resolution
pixel based classification methods present multiple challenges due to
the land surface spatial heterogeneity and the spectral similarity
between urban and non-urban materials (Myint et al., 2011). To
characterise the influence of spatial resolution on the ability to map
urban areas, high spatial resolution multispectral ortho-imagery (20 cm)
were classified into the four broad land cover types. To reduce data
processing requirements, four 3 km2 subsets were chosen that are
representative of the land cover composition and spatial heterogeneity
found within Perth (Figure 5 1 (a)). These subsets are an out of town
development area (East Beechboro), the Central Business District (CBD),
an older suburban area (Palmrya, Melville) and a largely vegetated
region (Keysbrook). Using the high spatial resolution multispectral
imagery and a relative elevation model, an Object Based Image Analysis
(OBIA) method was applied to classify each subset into vegetation,
urban, bare earth and water (Figure 5 3 (b)). OBIA methods are often
applied to high spatial resolution imagery as they include spatial,
textural and spectral information to classify the scene (Myint et al.,
2011). Incorporating surface elevation measurements into urban
classifications has been found to improve building (urban) extraction
accuracy (Aguilar et al., 2012; Poznanska et al., 2013). Surface
elevation estimates and Normalised Difference Vegetation Index (NDVI)
data provided additional urban classification parameters, with
refinement (e.g.~additions and alterations) made based on object
spatial, spectral and textural properties. Unlike the Landsat imagery,
the airborne imagery were collected during the late dry season when the
grass was senescent which resulted in textural and spectral similarity
between bare earth and roads. To mitigate the impact of potential
misclassification between these features, Landgate road and, where
appropriate, rail vector datasets were used for identification of
coincident image objects for urban assignment.

Table 5 1. The percentage of different land-cover types within the
classified high spatial resolution subsets (Figure 5 1). Subset
Vegetation (\%) Urban (\%) Bare earth (\%) Water (\%) East Beechboro
81.00 16.56 2.37 0.07 CBD 33.33 65.66 0.91 0.10 Palmrya 57.29 42.21 0.42
0.08 Keysbrook 97.36 0.90 1.56 0.18

\subsection{Dataset comparison and Landsat
refinement}\label{dataset-comparison-and-landsat-refinement}

In order to compare the orthophoto and Landsat land cover
classifications, the two urban (high and low reflectance) and two
vegetation (woodland and grassland) Landsat land cover classes were
merged so that both land cover classifications contained four identical
classes. To facilitate comparison between the high spatial resolution
orthophoto-derived classification and the Landsat classification, the
orthophoto land cover data is aggregated to Landsat spatial resolution
to provide a `soft' and a `hard' land cover dataset. To create the soft
30 m2 orthophoto-derived classification, each resampled 30 m2 pixel area
contains the proportion of each land cover type within it (Lu et al.,
2011) (Figure 5 3 (b)). This dataset was subsequently `hardened' by
assigning the pixel land cover type according to the dominant land cover
found within the 30 m2 area. The comparison methodology is to firstly
compare the per-pixel (i.e.~hardened) Landsat land cover classification
with the aggregated (30 m2) orthoimage classification. Misclassified
Landsat pixels are assessed further to establish the conditions that
lead to erroneous classification using the sub-pixel proportion
information (i.e.~soft classification datasets). The latter are also
used to identify a spatially explicit correction model to improve urban
area estimates from moderate spatial resolution imagery.

\section{Results}\label{results-1}

\subsection{Orthophoto and Landsat land cover
comparison}\label{orthophoto-and-landsat-land-cover-comparison}

A comparison is conducted between the orthophoto land cover
classification, aggregated to 30 m2 spatial resolution using the
majority land cover, and the IVM `hardened' Landsat classification. At
its native spatial resolution (20 cm; Figure 5 4 ((a-d)(i))), the
orthophoto land cover classification (Figure 5 4 ((a-d)(ii))) captures
the land cover spatial heterogeneity found within each region and
highlights the difference in the spatial structure between these
regions.

Figure 5 4. (i) High spatial resolution true colour orthophotos, (ii)
land-cover maps, and (iii) the agreement between the orthophoto
classification resampled to 30 m2 and the Landsat classification for:
(a) an out of town development area (East Beechboro), (b) old inner city
urban area (central business district), (c) older suburban area
(Palmrya, Melville), and (d) predominantly vegetated site (Keysbrook).
In (iii), areas depicted as `true' indicate those 30 m2 pixels where the
orthophoto land-cover type, based on the dominant land cover in the 30
m2 area, and Landsat land-cover type are in agreement. A comparison is
carried out between the orthophoto land cover classification, aggregated
to 30 m2 spatial resolution, and the `hardened' Landsat classification.
Figure 5 4 (iii) illustrates the spatial agreement between these
datasets and highlights those pixels where the same land cover type
(true) has been assigned to a pixel in both classifications. The areas
which are more homogeneous at Landsat's spatial resolution, such as the
CBD (urban, Figure 5 4 (b)) and Keysbrook (vegetation, Figure 5 4 (d)),
have greater level of agreement (73.14\% and 95.68\% respectively). In
contrast, the more heterogeneous subsets (East Beechboro and Palmrya,
Figure 5 4 (a and c)), have much lower levels of agreement (56.09\% and
32.03\% respectively). The differences in agreement result from the
sub-pixel heterogeneity at 30 m2 spatial resolution. Table 5 2 shows the
percentage of Landsat pixels which contain \textgreater{}50\% of a given
land-cover for each subset region.

Table 5 2. The percentage of pixels which contain \textgreater{}50\% of
a given land-cover type in each region. Subset Vegetation (\%) Urban
(\%) Bare earth (\%) Water (\%) East Beechboro 87.57 9.84 1.89 0.06 CBD
26.14 72.81 0.74 0.05 Palmrya 66.71 32.33 0.21 0.07 Keysbrook 98.90 0.05
0.88 0.11

To investigate the influence of sub-pixel heterogeneity on the ability
of Landsat to identify the pixel land cover type, the classification
accuracy is determined as a function of the percentage of urban area
within each Landsat pixel for all four subsets (Figure 5 5). The urban
percentage cover within each Landsat pixel is derived from the
orthophoto land cover classification which has been aggregated to 30 m2
and which provides the proportion of each land cover within each pixel.
The accuracy of the hardened Landsat classification was determined
through comparison against the `hardened' (e.g.~aggregated to 30 m2)
orthophoto land cover classification where the per-pixel land cover type
was determined based on the land cover type with the greatest sub-pixel
proportion. Figure 5 5 indicates that the hardened Landsat
classification results in a relatively high accuracy, with an average of
85.40\% (excluding Keysbrook), for pixels containing \textgreater{}50\%
urban land cover (according to the high spatial resolution land cover
classification). In the subsets of East Beechboro, the CBD and Palmrya,
the overall Landsat classification accuracy drastically declines to
1.99-6.21\% when urban land cover within a 30 m2 pixel area decreases to
40-50\%. The classification accuracy then increases with decreasing
sub-pixel urban cover which is particularly evident with Landsat pixels
containing 0-10\% urban cover. Keysbrook, on the other hand, is a
largely vegetated region and exhibits lower accuracy with increasing
urban land cover.

Figure 5 5. Landsat classification accuracy as a function of the
percentage urban cover within Landsat image pixels (as derived from the
high spatial resolution land-cover data set) for each of the four
subsets. In the Keysbrook subset, no Landsat pixels contained
\textgreater{}60\% urban land cover.

In order to understand the counter-intuitive behaviour of such as rapid
decrease in classification accuracy in pixels which contain between
40-50\% urban area (Figure 5 5), an analysis of the percentage of pixels
classified as a given land cover type is presented. To do so, all pixels
containing different ranges in urban percentage cover (e.g.~0-10\%,
20-30\% etc) were identified using the high spatial resolution land
cover dataset. The total percentage of each land cover type was
calculated for all pixels that contained urban percentage cover within
each range urban percentage cover (e.g.~0-10\%, 20-30\% etc) using
hardened IVM Landsat land cover dataset and the aggregated high spatial
resolution land cover dataset (i.e.~defined by the dominant land cover
type within a 30 m2 pixel area).

Figure 5 6 illustrates the percentage of pixels identified as a given
land cover type as indicated by the hardened Landsat land cover dataset
and the hardened high spatial resolution orthophoto land cover dataset
for pixels which contain differing percentage urban cover (e.g.~0-10\%)
derived using the original high spatial resolution orthophoto land cover
classification for the East Beechboro subset. This area was selected as
it is an intermediate area in terms of land cover heterogeneity (Figure
5 2 and Figure 5 4 (a)). The results indicate that the hardened Landsat
classification consistently overestimates urban land cover when compared
to the `hardened' high spatial resolution classification which has been
aggregated to 30 m2 based on the dominate land cover within the Landsat
pixel area for pixels with 10-50\% urban defined by high resolution
data. Table 5 3 and Figure 5 7 illustrates the sub-pixel (30 m2)
percentage urban land cover for East Beechboro with the original
reflectance imagery for this area shown in Figure 5 2. The hardened high
spatial resolution land cover dataset (left bar in each plot (Figure 5
6)) indicates that pixels containing \textless{}50\% urban land cover
are largely dominated by vegetation. In contrast, Landsat largely
identifies these pixels as being either urban or vegetated to differing
extents and more correctly identifies pixels with 0-10\% urban land
cover as being predominantly vegetated. For example, pixels containing
40-50\% urban area are correctly identified as being vegetated (98.45\%
of pixels within this range) by the hardened high spatial resolution
land cover dataset since these pixels contain on average 54.72\%
vegetation, 44.83\% urban and 0.45\% bare earth. In contrast, the
hardened Landsat land cover dataset identifies 5.65\% of pixels
containing 40-50\% urban cover as being vegetation, 74.28\% being urban
and 20.07\% being bare earth. As the percentage of urban land cover
decreases, the overall accuracy of the hardened Landsat classification
increases due to the increase in Landsat vegetation cover which
increases from 5.65\% (40-50\% urban cover) to 75.41\% (0-10\% urban
cover). The results are similar for the other regional subsets. The
rapid decrease in accuracy between 40-50\% and 50-60\% (Figure 5 5)
appears extreme as the subset regions are dominated by vegetation and
urban land cover (Table 5 1) which results in the aggregated 30 m2
pixels being assigned to vegetation when the percentage urban cover is
\textless{}50\% (Figure 5 6 (a-e)) or urban when the percentage urban
cover is \textgreater{}50\% (Figure 5 6 (f-j)).

Figure 5 6. Land-cover type disaggregation for urban land cover
(according to the orthophoto imagery) Landsat pixels in East Beechboro.
The left axis indicates the total percentage cover of a given land-cover
type using all of the pixels within a given range of urban percentage
cover range for: (a) 0--10\%, (b) 10--20\%, (c) 20--30\%, (d) 30--40\%,
(e) 40--50\%, (f) 50--60\%, (g) 60--70\%, (h) 70--80\%, (i) 80--90\%,
and (j) 90--100\%. For each percentage urban land-cover graph, the left
bar illustrates the overall percentage of pixels from the hardened high
spatial resolution classification identified as a given land types
whilst the right bar indicates the percentage of hardened Landsat pixels
mapped as a given land-cover type. Table 5 3. Urban area estimates (km2)
from high spatial resolution orthophoto land cover data for each subset
and those from the corresponding hard and soft IVM Landsat
classification. The overestimation of urban area by the hardened Landsat
land cover classification is evident. Percent difference to high
resolution (\%) 111.69 21.62 81.42 266.26 Percentage cover of subset
area (\%) 35.06 79.85 76.55 3.30 Landsat urban area sub-pixel (km2) 3.12
6.78 4.90 0.39 Percent difference to high resolution (\%) 118.66 30.54
103.60 252.22 Percentage cover of subset area (\%) 36.21 85.71 85.94
3.17 Landsat urban area (km2) 3.22 7.28 5.50 0.28 Percentage cover of
subset area (\%) 16.56 65.66 42.21 0.90 High resolution urban area (km2)
1.47 5.58 2.70 0.08 Subset East Beechboro CBD Palmrya Keysbrook

Figure 5 7. Comparison of percentage urban area aggregated to 30 m2 from
high-resolution data (a) and IVM `soft' Landsat classification (b)
highlighting the (overestimation) between the high (c) and moderate (d)
spatial resolution estimates for the East Beechboro subset. The
classified high spatial resolution data are shown in (e) with the
moderate spatial resolution grid (30 m2) overlaid for context (e).

The results in Figure 5 6 suggest that the spectral data used to train
the IVM classification (discussed in section 5.5.2) contained spectrally
`mixed' pixels resulting in land cover type misclassification. To
investigate this, the spectral reflectance from Landsat pixels
containing 20-30\% urban cover for the Palmrya subset, which had the
lowest overall agreement and which were identified as being mostly
vegetated by the hardened high spatial resolution land cover dataset,
are extracted and compared to the spectral reflectance profiles used to
train the IVM classification algorithm. Figure 5 8 indicates that there
are strong similarities between the average spectral reflectance profile
used to train the IVM classification algorithm and the average spectral
profile of the misclassified pixels. This suggests that the IVM
classification algorithm is accurately representing the Landsat pixel
spectral reflectance properties but that the training data used to
develop the classification model contained a high proportion of mixed
pixels.

Figure 5 8. Average spectral reflectance profile for misclassified
pixels (red) from the Palmrya subset for pixels containing 20--30\%
urban cover compared to the average spectral reflectance profile of
pixels used to train the IVM classification algorithm (blue). For (a)
forest, (b) low urban reflectance, (c) high urban reflectance, and (d)
bare earth. The error bars show the standard deviation.

Pure (i.e.~homogeneous) pixels are conventionally selected to train
classification models (e.g.~Weng and Pu, 2013) but these are inherently
difficult to identify in urban areas owing to the multitude of land
covers within a Landsat pixel area. Using the high spatial resolution
classification, the percentage of pure pixels, defined here as those
containing between 90-100\% of a single land cover type, were identified
(Table 5 4). It is evident that some regions contain a high percentage
of pure pixels for a given land cover type, such as vegetation in
Keysbrook (92.05\%), but that other land cover types within a region
typically have much lower percentages of pure pixels. Pure urban pixels
are particularly limited in all subset regions. Whilst the CBD subset
obtains a high percentage of pure urban pixels (28.77\%) these are
predominately urban areas with high spectral reflectance
(e.g.~concrete), differing from subsets with urban areas which have
urban areas with both high and low spectral reflectance (e.g.~East
Beechboro; Figure 5 2).

Table 5 4. Percentage of `pure' pixels (defined here as comprising
90-100\% of given landcover within a Landsat pixel area) from the high
spatial resolution imagery. Subset Vegetation (\%) Urban (\%) Bare earth
(\%) Water (\%) East Beechboro 53.93 0.15 0.34 0.03 CBD 8.98 28.77 0.35
0.00 Palmrya 5.80 2.13 0.00 0.01 Keysbrook 92.05 0.00 0.00 0.00

\subsection{Comparison between Landsat and high spatial resolution
impervious surface
estimates}\label{comparison-between-landsat-and-high-spatial-resolution-impervious-surface-estimates}

Landsat data have been widely applied to map impervious surface area in
order to assess its effects on: urban growth dynamics (Masek et al.,
2000), the UHI effect (Hu et al., 2015) and surface run-off (Weng,
2001). Figure 5 6 indicates that the `hardened' Landsat IVM
classification overestimates urban land cover, particularly for pixels
containing \textless{}50\% urban area. The IVM classifier also provides
a `soft' land cover dataset that quantifies the sub-pixel land cover
proportions.

Here we investigate the utility of the sub-pixel Landsat urban land
cover estimates by comparing them to those derived from the high spatial
resolution land cover dataset (20 cm) which is used to provide the
actual land cover proportion within each 30 m2 pixel area. Urban area
estimates from each of the four subsets (Figure 5 1 (a)) were spatially
averaged over different size spatial windows (30 × 30 m, 90 × 90 m, 150
× 150 m and 210 × 210 m) in order to account for any errors resulting
from pixel heterogeneity, spatial misregistration, residual atmospheric
and BRDF effects and phenological differences (Ghimire et al., 2010; Ju
et al., 2012; Liang et al., 2001; Lu et al., 2011; Maiersperger et al.,
2013) that may increase the uncertainty in estimating land cover
proportions (Lu et al., 2011; Sexton et al., 2013). Comparison of
impervious surface proportions at 30 m2, for example the CBD subset
(Figure 5 9), reiterates the overestimation of urban area at 30 m2
spatial resolution, with a clustering of values toward the upper
percentage boundaries associated with lower urban area estimates from
the high spatial resolution classification. When neighbourhood averaging
is applied, the agreement in urban area typically improves with
increasing window size although the subset specific bias remains
consistent (Table 5 5). It is also evident that urban area is still
overestimated with decreasing urban sub-pixel proportion even when
utilising the sub-pixel IVM Landsat classification results.

Figure 5 9. Relationship between the sub-pixel urban area percentage
cover estimated from the IVM sub-pixel Landsat classification and the
high spatial resolution orthophoto classification in the Central
Business District (CBD) subset for (a) 30 × 30 m window, (b) 90 × 90 m
window, (c) 150 × 150 m window, and (d) 210 × 210 m window.

Table 5 5. Comparison between high (20 cm2) and moderate (30 m2) spatial
resolution sub-pixel impervious surface estimates considering differing
kernel sizes over four subsets (Figure 5 1) within the PMR.

Subset Kernel size (m) R2 Scatter Bias Root Mean Square Error (RMSE)
East Beechboro 3030 0.41* 26.65 18.68 32.54

\begin{verbatim}
9090   0.68*   16.95   18.66   25.21
150150 0.75*   14.11   18.71   23.44
210210 0.80*   12.52   18.74   22.54
\end{verbatim}

CBD 3030 0.26* 28.41 14.38 31.84 9090 0.53* 16.65 14.37 22.00 150150
0.61* 13.18 14.38 19.51 210210 0.66* 11.30 14.36 18.28 Palmrya 3030
0.04* 26.65 34.54 43.62 9090 0.16* 13.56 34.61 37.17 150150 0.19*
10.15 34.64 36.10 210210 0.17* 8.45 34.67 35.69 Keysbrook 3030 0.24*
11.85 2.51 12.11 9090 0.52* 7.47 2.51 7.88 150150 0.60* 5.89 2.50 6.40
210210 0.63* 4.98 2.50 5.57

\subsection{Refining Landsat estimations using high spatial resolution
data}\label{refining-landsat-estimations-using-high-spatial-resolution-data}

Sub-pixel land cover heterogeneity influences Landsat urban area
overestimation which must be considered in order to reduce the bias and
improve Landsat derived urban area estimation (Herold et al., 2002; Lu
et al., 2011; Schneider, 2012; Varshney and Rajesh, 2014). The
complexity and diversity of urban areas identified here from high
spatial resolution data, with biases ranging from -2.50\% to -34.67\%,
highlights the inappropriateness of applying a single model to adjust
the moderate spatial resolution urban area estimates in a metropolitan
region (e.g.~Lu et al., 2011). The Landsat sub-pixel urban areas
estimates from all four subsets were stratified based on the Landsat
sub-pixel derived urban area and calibrated against the percentage of
urban area from the high spatial resolution classification within each
moderate spatial resolution pixel area. Both datasets were averaged at
the neighbourhood level using a 210 × 210m window as this provided the
best overall relationship (Table 5 5). Stratification of Landsat
sub-pixel urban estimates into divisions of 10\%, consistent with
previous results, were selected to develop (using 50\% of the data) and
test (remaining 50\% of the data) regression models to improve the
dataset agreement (Lu et al., 2011). The applied spatially explicit
models reduced the bias and Root Mean Square Error (RMSE) between the
predicted (moderate spatial resolution) and observed (high spatial
resolution) estimates (Table 5 6). It is evident from Table 5 6 that the
adjustment made to the Landsat urban area estimates reduced the
overestimation difference of urban area by between 34.38\% and 80.67\%,
with the largest improvement found within Keysbrook. Whilst the
corrected Landsat urban area estimates still overestimates the urban
area compared to the high spatial resolution dataset the corrected
moderate spatial resolution urban area reduces moderate resolution urban
area over (under) estimation by on average 55.08\% in comparison to the
high spatial resolution dataset reducing the average overestimation from
11.86 km2 per subset to just 0.09 km2 (Table 5 6). In the case of this
study area, this approach is appropriate for producing more accurate
urban area statistics. Due to the frequently reported over and under
estimation of land cover estimates by moderate spatial resolution data
this approach can refine urban estimates for planning development
policies that may inform decision makers (Hepinstall-Cymerman et al.,
2013; Schneider et al., 2005; Zhu and Woodcock, 2014). However, the
derived correction values are not globally applicable since the spatial
structure and makeup of urban and suburban areas varies regionally,
nationally and globally. Nevertheless the methodology implemented here
could be replicated to produce localised correction values from other
sources of high resolution imagery (e.g.~digitisation of Google Earth
imagery) to calibrate urban area estimates from moderate spatial
resolution data.

Table 5 6. Comparison between calibrated moderate (30 m2) and high (20
cm2) resolution sub-pixel impervious surface estimates with a kernel
size of 210m. * = statistically significant relationship
(p\textless{}0.05). Corrected percent difference to high resolution (\%)
8.83 -11.93 15.71 35.29 Corrected Landsat urban (km2) 2.72 10.88 8.10
0.19 Uncorrected percent difference to high resolution (\%) 72.47 22.45
57.32 115.96 Uncorrected Landsat urban (km2) 5.32 15.36 12.48 0.50 High
resolution urban (km2) 2.49 12.26 6.92 0.13 Root Mean Square Error
(RMSE) 6.76 14.61 12.53 1.38 Bias 1.54 -7.12 7.43 0.37 R2 0.84* 0.52*
0.12* 0.62* Subset East Beechboro CBD Palmrya Keysbrook

\section{Discussion}\label{discussion-1}

Refined urban estimates are vital in ensuring suitable sustainable and
strategic planning decisions are implemented (Bettencourt and West,
2010; Wu and Murray, 2003). The hybrid spatial resolution approach
applied here to estimate urban area was necessary due to the difficulty
in accurately estimating urban area using a traditional per-pixel
classification methods. This was due to a combination of the sensors
moderate (30 m2) spatial resolution, land surface heterogeneity and the
selection of `mixed' pixels for use in training the classification
algorithm. The overall classification accuracy, determined using Google
Earth imagery, was on average 84.00\%, which is similar to that found in
other studies, albeit for different urban areas (e.g.~Bagan and
Yamagata, 2014; Gislason et al., 2006; Luo et al., 2014; Sundarakumar et
al., 2012). Closer examination of the moderate spatial resolution
classification results using a higher resolution dataset indicates that
when urban land cover within a 30 m2 area decreases to 40-50\% (based on
high spatial resolution classification) the Landsat classification
accuracy decreased from 85.40\% to between 1.99 and 6.21\%. This
resulted from the Landsat classification overestimating urban area in
comparison to high spatial resolution data (Figure 5 5) which more
correctly identified these pixels as containing a greater per-pixel
proportion of vegetation. Pixels containing 40-50\% urban cover,
contained on average 54.50\% vegetation cover excluding Keysbrook. The
dominance of vegetation and urban land covers in the regional subset,
when ascribed to a 30 m2 pixel area based on the majority land cover,
results in a rapid change in classification accuracy. Strong spectral
similarities between training data and misclassified pixels (Figure 5 8)
suggests that the spectral reflectance observations used to train the
classification algorithm contained spectrally mixed pixels. The average
percentage urban area within a moderate spatial resolution pixel area
derived from the high resolution data was 16.56\%, 65.66\%, 42.21\% and
0.90\% for East Beechboro, CBD, Palmyra and Keysbrook respectively. The
percentage of `pure' pixels, defined as those containing over 90\% urban
land cover, was 28.77\% for the CBD but \textless{}2.50\% for the
suburban regional subsets. This highlights the difficulty in selecting
pure pixels at moderate spatial resolution and in accurately
disentangling mixed spectral reflectance's without the aid of high
spatial resolution data. Overestimation of urban extent was most
prominent in Keysbrook, where vegetation dominates the subset (97.36\%,
Table 5 1). In this instance, Landsat derived urban area corresponded to
0.28 km2 compared to 0.08 km2 from high spatial resolution
classification; a difference of only 0.20 km2 but which equates to
251.74\%. In terms of total area difference, the East Beechboro and the
CBD Landsat subsets were found to contain 1.75 km2 and 1.70 km2 more
urbanised area, whilst Palmyra data overestimated urban area by 2.79 km2
compared to the high spatial resolution equivalent due to its suburban
nature and associated pixel heterogeneity (Figure 5 4).

Spatially averaging the Landsat and orthophoto land cover
classifications, to account for potential errors in the datasets
(Ghimire et al., 2010), improved their relationship although Landsat
still overestimated urban area with differing bias per subset. Over
(under) estimation of urban land from Landsat estimations could result
in an under (over) prediction on further environmental variables
(e.g.~UHI) or policy applications. Multiple studies have used classified
per-pixel moderate spatial resolution data to influence policy changes
through monitoring urban growth (e.g.~Hepinstall-Cymerman et al., 2013;
Schneider et al., 2005). However, per-pixel methodologies fail to
address the issue of mixed pixels, which, as shown here, can result in
overestimation of urban area (average: 126.25\%, equivalent to 57.58 km2
within the PMR) (Lu et al., 2011). Sub-pixel methods attempt to remedy
this issue, but have been found to inaccurately separate impervious land
cover from other land cover types resulting in poor representation of
impervious surface area (Lu et al., 2011). Consequently over estimation
of urban area may have resulted in sub-optimal policies that fail to
maximise resource and amenity efficiency (Downs, 2005; Turner et al.,
2010).

Calibrating Landsat urban estimates using high spatial resolution data
reduces the bias, RMSE and improves urban area estimation. However, the
range of bias values across subsets of differing urban land cover
characteristic highlights the inappropriateness of a single regression
model due to pixel heterogeneity influencing overestimation (Lu et al.,
2011). Spatially explicit models, as presented here, permit varying
moderate spatial resolution refinement by considering the influence of
surface heterogeneity. Whilst the limited availability of low cost high
spatial resolution data can preclude analysis of this type, subset
digitisation of Google Earth or Unmanned Aerial Vehicle (UAV) imagery
may provide a suitable alternative for calibrating Landsat data for
improved urban area estimates. Enhanced estimates of urban area would
facilitate planning policies which avoid potential environmental and
socio-economic consequences of urban development than can result from
policies based on over (or under) predicted urban area (ARUP and The
Rockefeller Foundation, 2015). For example, classified Landsat data was
used to identify spatial clustering, peri urban development and
specialisation of land use in Chengdu, Sichuan province not considered
by China's original 1990 Go West policy, aimed at economically boosting
the West of the country. Results were used to reform policy and
remediate issues of urban management including: service, infrastructure
and resource deficiencies (Schneider et al., 2005). However, traditional
Landsat classification may over (or under) estimate urban area and
result in ineffective planning, environmental and policy decisions
(Miller and Small, 2003; Pravitasari et al., 2015). Therefore classified
sub-pixel data alongside high spatial resolution imagery (e.g.~UAV,
Google Earth, high spatial resolution aerial or satellite imagery) as
presented here can refine urban estimates facilitating improved decision
making whilst maximising often limited financial resources. This is
especially important in developing countries in regards to directing
urban development and resources based on factors including: poverty,
environmental hazards (e.g.~flooding) and current amenity centres
(Marfai et al., 2014; Suryahadi and Sumarto, 2003).

\section{Conclusion}\label{conclusion}

Landsat imagery from 2007 was used to map the urban extent within the
PMR using an IVM classifier which provides both a per-pixel and a
sub-pixel classified datasets. The 2007 Landsat classification overall
average accuracy was 84.00\% with associated Kappa coefficient of 0.78.
Comparison between the Landsat per-pixel urban area and urban area
estimates obtained from a high spatial resolution (20 cm)
orthophoto-derived classification indicates that the moderate spatial
resolution classification overestimates urban extent by 126.25 \% on
average, which is equivalent to 57.58 km2 in the study area. Similarly,
when the high spatial resolution urban area estimates are compared to
those derived using a sub-pixel Landsat classification, the latter still
overestimates urban extent by 120.25\%.

Accurately quantifying urban expansion within the PMR due to the large
population growth over the last decade is important in order to make the
efficient use of current resources and to avoid additional amenity,
environmental and health expenditure that can impact sprawling cities.
Landsat data provides the longest time series of medium spatial
resolution imagery to map and monitor urban area. However, the reported
over and underestimation inhibits accurate quantification of urbanised
land cover which increases uncertainty within global climate models,
environmental studies and targeted urban planning policy. Neighbourhood
averaging, to account for potential errors in the datasets, improved the
agreement between the two datasets but Landsat sub-pixel overestimation
still remained. The broad differences in bias between the difference
subsets indicates that a single regression model is inappropriate to
heterogeneous urban land cover estimates. Therefore, the moderate
spatial resolution urban area estimates were corrected using spatially
explicit regression models which, on average, across the four subsets
reduced the bias and RMSE by 17.02 km2 and 6.65 km2 respectively, whilst
reducing moderate resolution urban area over (under) estimation by
55.08\% Current and future EO satellites that provide complimentary data
with enhanced spatial, spectral and temporal resolution, such as
Sentinel-2, may further reduce over or under estimation of urban area
experienced by moderate spatial resolution sensors such as Landsat.
Similarly, high spatial resolution satellite sensors, such as
Worldview-3, are able to remediate discrepancies by capturing the fine
spatial detail of urban environments but their cost and small swath
limit their widespread application. This might change with companies,
such as Planet, which are launching large numbers of small
micro-satellites that provide high spatial resolution data more
frequently. Accurate urban land cover and land use mapping is essential
in understanding the impact of urban expansion on, for example,
social-ecological systems and human-health and will improve future
sustainable planning of our cities.

\section{Supplementary material}\label{supplementary-material-1}

This supplementary material sections supports the main chapter and
thesis through providing a review surrounding classifier selection,
pertinent to the selection of the Import Vector Machine (IVM) classifier
implemented in generating the classified Landsat image used within the
chapter. A short discussion surrounding other considered approaches and
a list of data used within the chapter are also provided to demonstrate
the entirety of analytical process.

\subsection{Classification
methodologies}\label{classification-methodologies}

Currently two main image analysis techniques exist for urban mapping:
spectral indices and classification algorithms. Spectral indices such as
the Normalised Difference Built-up Index (NDBI) have been used to
delineate urban areas from non-urban (Zha et al., 2003). NDBI identifies
built up regions using a ratio of the shortwave-infrared (SWIR) and near
infrared (NIR) wavebands and assumes that built up areas have higher
SWIR reflectance (Xu, 2008). In comparison to ground truth observations,
NDBI-derived classification from Landsat Thematic Mapper (TM) over
Nanjing, China was found to result in an overall accuracy of 92.6\% (Zha
et al., 2003). However, due to the heterogeneous nature of urban
environments, the identification of built up areas by thresholding a
spectral index is not always reliable (Xu, 2008).

The Impervious Build-up Index (IBI) attempts to mitigate for this by
using a combination of a number of thematic indices namely: NDBI, the
Soil Adjusted Vegetation Index (SAVI) and the Modified Normalised
Difference Water Index (MNDWI) (Xu, 2008). The index amplifies the
identification of built up land through the inclusion of ancillary
information on the presence of bare surfaces (SAVI) and water bodies
(MNDWI) resulting in positive values for pixels identified as being
urban (Xu, 2008). Nevertheless, urban areas often remain an inseperable
mix of impervious and bare earth surfaces which require additional
post-processing to delineate (Stathakis et al., 2012; Sun et al., 2015;
Zha et al., 2003).

In the second instance, classification algorithms are defined as
parametric (e.g.~Maximum Likelihood (ML)) or nonparametric
(e.g.~Decision Trees (DT)), depending on whether training samples can be
represented by a Gaussian probability density function (Donnay and
Unwin, 2001; Jensen, 2005). Maximum Likelihood accounts for the
variance-covariance within class distributions and has been implemented
for monitoring land cover change and to derive sub-pixel proportions
(Atkinson et al., 1997; Shalaby and Tateishi, 2007). However, due to the
parametric assumption of multivariate normal data, the ML classifier can
often fail to represent land cover that might be multimodal (Melgani and
Bruzzone, 2004; Mountrakis et al., 2011; Otukei and Blaschke, 2010). An
example of this issue is illustrated in semi-arid locations, such as
grasslands, which are sensitive to precipitation timing and volume that
can result in differing multimodal spectral-temporal profiles (Friedl et
al., 2002). A decision tree methodology was utilised to generate the
United States of America National Land Cover Database 2001 (NLCD 2001)
resulting in a nonparametric approach able to handle continuous and
nominal data, interpretable classification rules and swift application
(Homer et al., 2004). Nevertheless, DTs can be negatively affected by
pruning methods, for example Pessimistic Error Pruning (PEP) introduces
a continuity correction value, within error estimation on no theoretical
basis, resulting in under or over pruning (Esposito et al., 1997; Otukei
and Blaschke, 2010; Pal and Mather, 2003).

More recently, Machine Learning Algorithms (MLA) or `expert systems'
(e.g.~Support Vector Machine (SVM)) have been implemented for image
classification (Jensen, 2005; Okujeni et al., 2014) using an automated
inductive approach for identification of patterns in data (Cracknell and
Reading, 2014). SVM is a nonparametric binary statistical learning
methodology that separates a dataset into example classes (training
data) based on a decision boundary, or hyperplane, with an aim to
minimise misclassification. The optimal maximum margin separating
hyperplane divides the data into a predefined number of classes, with
points on the margins termed `support vectors' (Foody and Mathur, 2006,
2004). The underlying benefit of SVM pertains to structural risk
minimisation, whereby SVMs are able to minimise error on unseen data
without prior assumptions on the distribution (Mountrakis et al., 2011;
Vapnik and Chervonenkis, 1971). SVMs are linear binary classifiers
which, when deriving more than two classes, require implementation of an
additional process, either a one-against-all or one-against-one
analysis. One-against-all solves for the multiple optimisation problem,
which separates one class from the remaining classes. Comparatively
one-against-one combines multiple classifiers and performs pair-wise
comparisons using a `voting' process to assign a pixel to a land cover
class, based on the class assigned the most votes (Chih-Wei et al.,
2008; Mountrakis et al., 2011; Pal and Mather, 2005). Within SVMs
implementation of soft margin and kernel methods aid separability
through the introduction of additional variables that ignore hyperplane
outliers and transform data into high dimensional feature spaces
(Euclidean or Hilbert) utilising non-linear functions to identify linear
solutions respectively (Braun et al., 2012; Cortes and Vapnik, 1995;
Melgani and Bruzzone, 2004; Mountrakis et al., 2011).

SVMs have been extensively used for classification purposes, due to
their ability to ignore inherent image errors and to avoid overfitting
(Foody and Mathur, 2006; Mountrakis et al., 2011). SVMs have obtained
broad applicability for land cover classification using data from a
multitude of sensors such as HyMAP (Camps-Valls et al., 2004), Advanced
Spaceborne Thermal Emission and Reflection Radiometer (ASTER) (Zhu and
Blumberg, 2002) and Landsat (Knorn et al., 2009) producing
classification results with accuracies between 85\% and 95\%.

\subsection{Other considered data and
approaches}\label{other-considered-data-and-approaches-1}

In proposing the research within this chapter multiple other datasets
were considered in replace of Landsat. These included: GlobeLand30
(classified Landsat data for 2000 and 2010) (Chen et al., 2015), the
European Space Agency's (ESA) Climate Change Initiative (CCI) annual
global land cover (300 m) time series (1992-2015) (European Space
Agency, 2017) and yearly MODIS land cover products (500 m or 0.05°)
(Friedl et al., 2002). Nevertheless these products are constrained by:
resolution, temporal frequency, collection dates to closely match the
high resolution imagery and global accuracy assessments that might fail
to represent the Perth Metropolitan Region. Additionally, using data
produced within chapter 4 enabled greater flow throughout the thesis.
Similarly, other datasets under consideration in replace of the high
resolution orthophotos included that from Google Earth (through
digitisation) and digital elevation data from Geoscience Australia (5
metre resolution) (Geoscience Australia, 2015). However, the potential
for analyst error alongside lower resolution elevation data could have
potentially exacerbated errors within the classification process.
Consequently, object based image analysis of the high resolution
orthophotos was selected as it provided a repeatable approach with
minimal subjectivity (Myint et al., 2011).

\subsection{Chapter data list}\label{chapter-data-list-1}

This section provides an overview of the data presented and analysed
within this chapter.

Supplementary Table 5 7. Summary of the data, sources and applications
used within chapter 5. Data Source Application Classified Landsat land
cover data (including sub pixel estimates) from 2007 Raw data
information (2 images) Landsat 5 TM, 6/10 (path 112), 9/07 (path 113),
row 82 Chapter 4 or (MacLachlan et al., 2017b) Comparison to high
resolution urban estimates High resolution (20cm) orthophotos comprised
of four spectral bands, digital surface and elevation models Sensor
Microsoft UltraCAM-D Collection dates 19 could free days from 14/03/2007
Australian Commonwealth Scientific and Industrial Research Organisation
Perth and Peel Urban Monitoring Programme To compare classified high
resolution data to that of Landsat in order to remediate Landsat's
overestimation Railway and road vector data Landgate Classification of
the high resolution imagery

\chapter{Urbanisation-induced land cover temperature dynamics for
sustainable future urban heat island
mitigation}\label{urbanisation-induced-land-cover-temperature-dynamics-for-sustainable-future-urban-heat-island-mitigation}

\section{Abstract}\label{abstract-3}

Urban land cover is one of the fastest global growing land cover types
which permanently alters land surface properties and atmospheric
interactions, often initiating an urban heat island effect. Urbanisation
comprises a number of land cover changes within metropolitan regions.
However, these complexities have been somewhat neglected in temperature
analysis studies of the urban heat island effect, whereby
over-simplification ignores the heterogeneity of urban surfaces and
associated land surface temperature dynamics. Accurate spatial
information pertaining to these land cover change -- temperature
relationships across space is essential for policy integration regarding
future sustainable city planning to mitigate urban heat impacts. Through
a multi-sensor approach, this research disentangles the complex spatial
heterogeneous variations between changes in land cover (Landsat data)
and land surface temperature (MODIS data), to understand the urban heat
island effect dynamics in greater detail for appropriate policy
integration. The application area is the rapidly expanding Perth
Metropolitan Region in Western Australia. Results indicate that land
cover change from forest to urban is associated with the greatest annual
daytime and nighttime temperature change of 0.40 °C and 0.88 °C
respectively. Conversely, change from grassland to urban minimises
temperature change at 0.16 °C and 0.77 °C for annual daytime and
nighttime temperature respectively. These findings are important to
consider for proposed developments of the city as such detail is not
currently considered in the urban growth plans for the Perth
Metropolitan Region. The novel intra-urban research approach presented
can be applied to other global metropolitan regions to facilitate future
transition towards sustainable cities, whereby urban heat impacts can be
better managed through optimised land use planning, moving cities
towards alignment with the 2030 sustainable development goals and the
City Resilience Framework.

\bibliography{book.bib}

\end{document}
